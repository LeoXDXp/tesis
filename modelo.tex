\chapter{Algoritmo Propuesto}
\label{sec:modeloOpti}
%dado que se tienen muchos filamentos, se debe evaluar cual es mejor. Explicar como propiedades topológicas y geométricas tienen. Y como se ponderan para un peso que será minimizado o maximizado

En base a lo recopilado en los cap\'itulos previos de esta investigaci\'on, es posible destacar los siguientes aspectos al problema a resolver:

\begin{itemize}
    \item Se desconoce a priori el n\'umero de filamentos a buscar, dado que una imagen puede tener individualizaciones distintas para 2 expertos.
    \item Generalmente, se busca individualizar m\'as de un filamento por imagen, lo que conlleva a elegir los mejores filamentos entre las soluciones que se encuentren.
    \item El uso de un grafo para representar la red de filamentos puede implicar que las combinaciones de soluciones crezcan de manera exponencial.
\end{itemize}

Lo anterior implica que el problema de identificar filamentos a partir de un grafo puede ser clasificado como un problema de optimizaci\'on de restricciones\cite{blum2011hybrid}.

Un problema de optimización de restricciones, (COP por su sigla en ingl\'es) puede ser representado como $P = (S, \Omega, F)$, donde S es el espacio de soluciones, $\Omega$ son las restricciones, y $F$ es la funci\'on objetivo. S esta definido por un conjunto discreto de variables $X = 1 \dotsc n$, con valores $v_{i}^{j} \in D_{i} = \{v_{i}^{1} \dotsc  v_{i}^{|D_{i}|}\}$. Se define como una variable {\it instanciada} la asignaci\'on a $X_i$ de un valor $v_{i}^{j} \in D_i$. Una solución candidata $s \in S$ es una soluci\'on factible si satisface las restricciones del conjunto $\Omega$. La funci\'on objetivo $F: S\rightarrow \mathbb R_{0}^{+}$, es la funci\'on de evaluaci\'on que asigna valores a las soluciones candidatas. Al mismo tiempo, se define $s^{*}$ como una soluci\'on \'optima y $S^{*}$ como el conjunto que engloba todas las soluciones \'optimas, relacionados mediante $s^{*} \in S^{*} \subseteq S $\cite{socha2008ant}.
%Esta definici\'on permite aplicar la metaheur\'istica de optimizaci\'on basada en colonia de hormigas (ACO por su sigla en ingl\'es) a un modelo de un {\it COP}.
%COP es un CSP con función objetivo: https://en.wikipedia.org/wiki/Constrained_optimization#Constraint_optimization_problems

Es posible adaptar la definici\'on del modelo COP al modelo de feromonas de la metaheur\'istica ACO. Esta metaheur\'istica se inspira en el comportamiento de una colonia de hormigas en la b\'usqueda del trayecto m\'as corto a la comida, comunic\'andose entre ellas mediante feromonas. El recorrido de una hormiga se define como un camino $s$ y representa una soluci\'on en la b\'usqueda del camino m\'as corto.


As\'i, en base a la definici\'on de variable {\it instanciada} del modelo COP, se tiene que la asignaci\'on $X_i = v_{i}^{j}$ es equivalente a seleccionar un componente de soluci\'on $c_{ij}$ para una soluci\'on o camino $s$ en ACO.
%\section{Metaheur\'istica ACO}
Una soluci\'on $s$ consiste en un conjunto de componentes de soluci\'on $c_{ij} \in C, i = 1 \dotsc n, j = 1 \dotsc |D_i|$, por lo que una concatenaci\'on de componentes de soluci\'on forma el camino o {\it tour} que recorre una hormiga. En el caso de un grafo, el recorrido es desde un nodo o arista inicial hasta un nodo o arista final. La metaheur\'istica ACO permite encontrar una soluci\'on o un conjunto de soluciones. 
 
La metaheur\'istica ACO, indicada en el algoritmo \ref{ACO-Algo}, consiste en un paso de inicializaci\'on y de tres componentes, las que no tienen un orden espec\'ifico: {\it Construccion\_de\_soluci\'on\_de\_cada\_hormiga(),  M\'etodo\_de\_b\'usqueda\_no\_local()} y {\it Actualizaci\'on\_de\_feromonas()}.


\begin{algorithm}[H]
\SetAlgoLined
\KwData{Variables $X_i \dotsc X_n$, dominios $D_1 \dotsc D_n$, Restricciones $\in \Omega$}
\KwResult{conjunto s\textquotesingle $ \subseteq S$ != $\emptyset$, si existen soluciones factibles}
 Ajuste de Par\'ametros \& inicializaci\'on de feromonas \;
 \While{Criterio de finalización no se cumple}{
   Planificaci\'on\_de\_Pasos\;{
   ~ Construccion\_de\_soluci\'on\_de\_cada\_hormiga()\;
   ~ M\'etodo\_de\_b\'usqueda\_no\_local() \% opcional (DaemonActions)\;
   ~ Actualizaci\'on\_de\_feromonas()\;
   }Fin\_Planificaci\'on\_de\_Pasos\;
 }
 \caption{Algoritmo Metaheur\'istica ACO}\label{ACO-Algo}
\end{algorithm}


\section{M\'etodo de construccion de soluci\'on de cada hormiga}
\label{subsubsec:antTourInit}
Al comenzar un recorrido, cada hormiga es asignada una arista de acuerdo a la heur\'istica de asignaci\'on inicial, la cual analiza hasta 3 situaciones, dependiendo del tipo de c\'elula:
\begin{enumerate}
\item La arista a asignar debe tener al menos uno de sus nodos con grado 1, indicando que es el inicio o final de una parte del grafo.

\item De no haber aristas con esas caracter\'isticas disponibles, se realiza una asignaci\'on inicial de una arista con uno de sus nodos con grado 2 o superior, siempre que uno de los nodos sea la uni\'on de esta arista con otra con la que conformen un \'angulo en el rango $]\theta, Max\_Angle]$. 
%$\theta$ es un umbral que define el \'angulo m\'aximo, en grados, bajo el que se considera que 2 aristas contiguas respetan la rectitud necesaria para formar parte del mismo filamento. $Max\_Angle$ es un umbral que define el \'angulo m\'aximo, en grados, por sobre el cual se descarta de forma absoluta que 2 aristas contiguas forman parte del mismo filamentos. Este rango delimita los pares de aristas que a priori no representan combinaciones que respetan el criterio de rectitud, pero cuya explicaci\'on puede encontrarse en variaciones inducidas durante la extracci\'on del grafo desde la imagen, por lo que es necesario incorporar la exploraci\'on de estos pares de aristas.

\item De no existir aristas con alg\'un nodo que cumpla con los dos criterios previos, es posible asignar una arista aleatoria siempre que esta no forme parte de un camino recorrido por otra hormiga que haya sido evaluado como de buena calidad.
\end{enumerate}

%a elecci\'on de un componente $c_{ij}$ por una hormiga durante la construcci\'on de un camino, se lleva a cabo mediante el c\'alculo de una probabilidad para cada componente $c_{ij}$ posible de elegir. Este conjunto de vecinos factibles se denomina $N(s^{P}) \subseteq C$. En la probabilidad de selecci\'on influye el camino ya escogido, denominado soluci\'on parcial $s^{P}$. 

% a diferencia de otros ACO, aca s^P != \emptyset al comienzo
Una vez asignada la primera arista seg\'un la heur\'istica previamente descrita, cada hormiga debe avanzar mediante la elecci\'on de nuevas aristas para a\~nadirlas a su recorrido. Este procedimiento corresponde a la probabilidad de elegir una arista o componente $c_{ij}$ a partir de un conjunto de aristas vecinas (conjunto $N(s^{P})$) dadas la aristas que ya pertenecen a la soluci\'on parcial de la hormiga ($s^P$) (ecuaci\'on \eqref{eq:antProbabilities}). La probabilidad de elecci\'on de una arista en $N(s^{P})$ depende del valor de la feromona que tiene asociada ($\tau_{ij}$), as\'i como de la heur\'istica miope que eval\'ua la desviaci\'on que esta arista generar\'ia en el camino ($\eta_{ij}$). 
%mediante la heur\'istica miope (ecuaci\'on \eqref{eq:heuristicaMiope}) que privilegia los candidatos que causen la menor desviaci\'on en la rectitud del camino. 
Se define que las aristas $c_{ij}$ que aportan con mayor probabilidad a la menor desviaci\'on son aquellos que en conjunto con el \'ultimo elemento elegido por la hormiga en ese punto ($c_{(i-1)j}$) forman un \'angulo en el rango $[0, \theta]$. El criterio de finalizaci\'on para la hormiga corresponde a que el conjunto $N(s^{P}) = \emptyset$.

%P(C_{ij} | s^{P}) = P_{n_{i},n_{j}} = P_{e_{x}}
\begin{equation}
P(c_{ij} | s^{P}) = \frac
        {\tau_{ij}^{\alpha} \cdot \eta_{ij}^{\beta}}
        {\sum\limits_{c_{ij}\in N(s^p)}{\tau_{ij}^{\alpha} \cdot \eta_{ij}^{\beta} } }, \forall c_{ij} \in N(s^{P})
\label{eq:antProbabilities}
\end{equation}

Otro aspecto de la ecuaci\'on \eqref{eq:heuristicaMiope} radica en la posibilidad de elecci\'on de aristas $c_{ij}$ que tienen un \'angulo en el rango $]\theta, \text{Max\_Angle}]$ con el componente de soluci\'on $c_{(i-1)j}$. Esto facilita la exploraci\'on de soluciones/caminos que de forma miope aparecen como de calidad no \'optima, y que para los efectos de este trabajo se denominan como de {\it calidad intermedia}. Esta evaluaci\'on consistente en disminuir la probabilidad de elecci\'on a medida que se incrementa la diferencia entre el \'angulo que forman $c_{ij}$ y $c_{(i-1)j}$, y la mitad de $\theta$.


%donde cada uno representa a una arista en esta investigaci\'on, y . Al momento de que la diferencia sea 90\textdegree, la probabilidad de asignaci\'on se reduce al 50\% de la probabilidad de un componente $c_{ij} \in [0, \theta]$.

\begin{equation}
    \eta_{ij} = 
        \begin{cases} 
        \text{Max\_Score si } \measuredangle(c_{ij}, c_{(i-1)j}) \in [0, \theta]\\[3ex]
        
        \text{Max\_Score} \cdot \left(1 - \dfrac{ \left| \measuredangle(c_{ij}, c_{(i-1)j}) - \frac{\theta}{2} \right|} {180} \right)  \text{ si } \measuredangle(c_{ij}, c_{(i-1)j}) \in \quad ]\theta, \text{Max\_Angle}].\\[3ex]
        
        \text{0 en otro caso;}
        \end{cases}
    \label{eq:heuristicaMiope}
\end{equation}

Con la finalidad de cuantificar la calidad de la soluci\'on construida por una hormiga, en cada selecci\'on de arista realizada, se suma el resultado de la heur\'istica miope $\eta_{ij}$ a la calidad del camino que la hormiga lleva hasta ese punto. Cada hormiga comienza con una calidad 0, que al finalizar su recorrido es dividida por el n\'umero de aristas menos 1, para normalizar. Se establece que para ser considerada una soluci\'on de buena calidad, la hormiga debe tener una calidad mayor o igual a $\frac{Max\_Score}{2}$.
    
\section{M\'etodo de b\'usqueda no local}
Una vez que la hormiga termina un camino o {\it tour}, es posible agregar un m\'etodo de retroalimentaci\'on sobre la calidad del recorrido realizado, basado en l\'ogicas globales/centralizadas que escapan de la b\'usqueda local que realiza cada hormiga. Estos m\'etodos, denominados {\it Daemon Actions} en ingl\'es, permiten en el caso de una metaheur\'istica ACO gen\'erica seleccionar las hormigas de mejor calidad para incrementar las feromonas m\'as alla de lo que la {\it Actualizaci\'on\_de\_feromonas()} lo hace. 

Para la individualizaci\'on de filamentos, la evaluaci\'on global corresponde a eliminar soluciones candidatas que no aporten informaci\'on nueva. A modo de ejemplo, si $s_a$ y $s_b$ son las soluciones de las hormigas $a$ y $b$ respectivamente y cumplen con las siguientes condiciones:

\begin{itemize}
    \item $\forall c_{ij} \in s_a \in [0, \theta]$ y $\forall c_{ij} \in s_b \in [0, \theta]$
    \item $\forall c_{ij} \in s_a$ fueron electos por la hormiga $a$ con $P(c_{ij} | s_{a}^{P}) = 1$ y $\forall c_{ij} \in s_b$ fueron electos por la hormiga $b$ con $P(c_{ij} | s_{b}^{P}) = 1$
    \item $s_a \subseteq s_b$
\end{itemize}

Se tiene que $s_a$ no aporta m\'as informaci\'on que $s_b$, por lo que $s_a$ puede descartarse. Se denomina a $s_b$ como un segmento, el cual se comporta como una secci\'on indivisible de filamento. Este m\'etodo se ejecuta al comparar dos soluciones 
%RIESGO ASOCIADO a overmatch!!

%Si dos soluciones, $s_i$ y $s_j$ de las hormigas $i$,$j$, conformadas solamente por componentes $c_{ij} \in [0, \theta]$  (todos los componentes son de {\it buena calidad}), y que adem\'as eran la \'unica opci\'on posible en cada avance de la hormiga (probabilidad 1 de ser elegidas)
%tal que $s_i \subseteq s_j$ o viceversa, se tiene que una soluci\'on candidata no aporta nueva informaci\'on.
    
\section{M\'etodo Actualizaci\'on de feromonas}
\label{subsec:pheroUpdate}
Una vez que la hormiga termina un {\it tour}, esta debe actualizar las feromonas $\tau_{ij}$ asociadas a los componentes de soluci\'on que la conforman. En una metaheur\'istica ACO tradicional, se aumenta el valor en los $c_{ij}$ que construyen un camino de buena calidad, mientras que debe realizar lo contrario para los componentes de soluci\'on que son parte de un recorrido de mala calidad. Adem\'as, los valores de las feromonas sufren decaimiento en el tiempo, dado por el par\'ametro $\rho$, que busca evitar la convergencia que las feromonas pueden causar en caminos de buena soluci\'on obtenidos al inicio de las iteraciones.


En la individualizaci\'on de filamentos, se utilizan {\it anti-feromonas}, con el prop\'osito de indicar a las hormigas de futuras iteraciones cuales combinaciones de aristas/componentes de soluci\'n no llevan a resultados de buena calidad, con la finalidad de acotar el espacio de soluciones $S$. La forma de usar las {\it anti-feromonas} corresponde a {\it Substractive Anti-Pheromone} (SAP por su sigla en ingl\'es), la que introduce el par\'ametro $\gamma$ como factor de reducci\'on/penalizaci\'on. \cite{montgomery2002anti} indica que SAP con $\gamma = 0.5$ obtiene los mejores resultados. Por otra parte, el par\'ametro $\rho$ utilizado en las feromonas tradicionales no se utiliza en SAP.


Una modificaci\'on que se introduce en este trabajo con respecto a la aplicaci\'on de  feromonas y anti-feromonas es que estas com\'unmnente s\'olo se encuentran asociadas al componente $c_{ij}$ respectivo. Como variaci\'on para la individualizaci\'on/reconocimiento de filamentos, se propone asociar el valor de la anti-feromona no s\'olo con el componente $c_{ij}$, sino que adem\'as con uno o m\'as de los elementos que fueron elegidos en pasos anteriores por la hormiga que los selecciona ($s^{P}$). Se define como un segmento $seg_n$ de un camino al conjunto de aristas contiguas $c_{ij} \in s^{P}$ donde cada arista forma un \'angulo en el rango $[0, \theta]$ con la arista que la antecede. As\'i es posible desglosar una soluci\'on $s$ en segmentos $seg_{n}$ donde $n$ se\~nala el n\'umero de segmento al que corresponde dentro de la soluci\'on $s$. El segmento $seg_1$ comienza con la primera arista/componente $c_{ij}$ asignada a la hormiga de acuerdo a la heur\'istica de asignaci\'on inicial, y termina en el primer elemento $c_{ij}$ con el que forma un \'angulo de calidad intermedia (sin incluirlo), siendo este mismo elemento el que inicia el segmento siguiente.
%Existen $N + 1$ segmentos en $s$ si la soluci\'on contiene $N$ elementos $c_{ij} \in ]\theta, 90]$ (de calidad intermedia). 


Luego, mediante la anti-feromona se relaciona un segmento $seg_n \subset s$ y la componente de soluci\'on $c_{ij} \in seg_{n+1} \in s$, donde la elecci\'on de la arista o componente $c_{ij}$ despu\'es de haber elegido las aristas pertenecientes a $seg_n$ origina una soluci\'on de mala calidad. El rol de la anti-feromona es penalizar la probabilidad de elecci\'on de $c_{ij}$ para las hormigas que tengan al segmento $seg_n$ en su soluci\'on parcial $s^{P}$, con el fin de evitar que otras hormigas que pasen por $seg_n$ elijan $c_{ij}$.


Se debe destacar que esto es necesario ya que si s\'olo se utiliza la anti-feromona para penalizar la arista o componente $c_{ij}$, se puede ocasionar la perdida de capacidad de exploraci\'on de hormigas que provengan de otros recorridos parciales distintos a $seg_n$. La perdida de exploraci\'on tambi\'en sucede en el caso que el segmento $seg_{n+1}$ contenga solo 1 arista y no sea el segmento en el que finaliza el recorrido de la hormiga. En aquel caso, la \'unica arista de $seg_{n+1}$ se transforma en un cuello de botella para la exploraci\'on debido a que penaliza de forma indiscriminada a todas las hormigas que pasen por ah\'i, independientemente del \'angulo que forme con otras aristas o componentes. Para subsanar aquel caso, a este tipo de segmentos se le a\~naden los nodos del segmento que lo precede, $seg_{n}$, con el objetivo de evitar que el an\'alisis del par $seg_{n+1}$ con la componente $c_{ij}$ que inicia el segmento $n+2$ sea s\'olo de 2 aristas.


Adicionalmente a lo anterior, se ha agregado un l\'imite de 2 penalizaciones como m\'aximo para cada anti-feromona. Al alcanzar este l\'imite, se reduce el valor de la anti-feromona a 0, haciendo imposible la elecci\'on de la componente de soluci\'on $c_{ij}$ para las hormigas cuya soluci\'on parcial $s^{P}$ contenga el segmento en el par $<$componente,segmento$>$ penalizado. La anti-feromona se aplica sobre hormigas que han finalizado su recorrido y cuya calidad normalizada sea menor a {\it Max\_Score}, ya que esto implica que al menos 1 de las aristas del recorrido es de calidad intermedia, necesitando un an\'alisis adicional para determinar si corresponde a una soluci\'on de buena calidad. 


El an\'alisis adicional se separa en evaluaciones comunes que no dependen de la c\'elula observada, agregando posteriormente las evaluaciones particulares. El conjunto de evaluaciones comunes consisten en evaluar la curvatura del recorrido, como tambi\'en la magnitud del desplazamiento entre la proyecci\'on de un segmento en relaci\'on a otro segmento contiguo. La evaluaci\'on particular se acciona si la c\'elula es una neurona, debido a que se requiere verificar que la finalizaci\'on del recorrido no sea en una arista que cumpla con el primer criterio de la heur\'istica de asignaci\'on de aristas iniciales, descrita en la secci\'on \ref{subsubsec:antTourInit}.

La curvatura de recorrido de una hormiga $a$ es el \'angulo formado por el nodo inicial ($n_{a1}$), el centro de masa de la misma hormiga ($mc_{a}$) y el nodo final ($n_{af}$). Este \'angulo no debe superar el umbral definido al multiplicar el \'angulo $\theta$ por un factor denominado {\it Max\_Axial\_Displacement}. Este factor permite flexibilizar la tolerancia de la curvatura en base a $\theta$. Si el recorrido de la hormiga tiene un \'angulo igual o mayor al umbral, implica que la soluci\'on encontrada es demasiado curva para representar un filamento, por lo que se penaliza el par $\langle c_{ij}$,$ seg_{n}\rangle$ donde $c_{ij}$ es la componente de soluci\'on/arista que inicia el \'ultimo segmento de la hormiga, mientras que $seg_{n}$ es el segmento que lo precede. Posterior a la penalizaci\'on, se desecha la soluci\'on. El criterio de curvatura se refleja en la ecuaci\'on \eqref{eq:antiPheroSAP_Angle}.
%\begin{equation}
%    \label{eq:antiPheroSAP_Angle}
%    \tau_{ij} \leftarrow \tau_{ij} \cdot \gamma \quad \forall \langle c_{ij},seg_{n}\rangle > \textrm{Max\_Axial\_Displacement}
%\end{equation}

\begin{equation}
    \tau_{ij} \leftarrow
        \begin{cases}
        \tau_{ij} \cdot \gamma \text{ si } \measuredangle((n_{a1}, mc_{a}), (mc_{a}, n_{af})) < \theta \cdot \text{Max\_Axial\_Displacement}\\[3ex]
        
        \text{0 si } \tau_{ij} \leq 0.25 \\[3ex]
        \tau_{ij} \quad \text{en otro caso}
        \end{cases}
    \label{eq:antiPheroSAP_Angle}
\end{equation}

%agregar referencia? tindemans rod straightness \cite{hawkins2010model} of MTs o 
El an\'alisis respecto a la magnitud del desplazamiento entre la proyecci\'on de un segmento en relaci\'on a los segmentos que lo preceden se fundamenta en la rigidez que algunos tipos de filamentos como los microt\'ubulos y los filamentos de actina poseen\cite{stam2017filament}. El criterio de rigidez consiste en analizar los segmentos con respecto a la totalidad de sus predecesores, comenzando por el \'ultimo segmento recorrido por la hormiga, es decir, desde el extremo final de la soluci\'on. A partir de cada par de $<$segmento,segmentos$_$predecesores$>$, $\langle seg_{n}$,$seg_{1,n-1}\rangle$, se debe seleccionar el miembro del par de mayor longitud, definido como $s_{max} = \max(\norm{seg_{n}}, \norm{seg_{1,n-1}})$, para calcular el \'angulo suplementario que forma con respecto al otro miembro del par, definido como $s_{min}$. El \'angulo suplementario, $\measuredangle supl(s_{max},s_{min})$ es el equivalente a calcular el \'angulo entre la proyecci\'on de $s_{max}$ y $s_{min}$.

Luego $s_{min}$ es multiplicado por el seno del \'angulo suplementario, estableciendo el desplazamiento con respecto al eje que forma $s_{max}$ y su proyecci\'on. El umbral que delimita al desplazamiento calculado previamente se define como el m\'ultiplo de $s_{max}$ por 10\% de {\it Max\_Axial\_Displacement}. 

Si el \'angulo suplementario es menor a $\theta$, se puede declarar que se cumple el criterio de rigidez. Lo anterior se refleja en la ecuaci\'on \eqref{eq:antiPheroSAP_Axial}. 


%Esta propiedad puede ser utilizada para delimitar como un segmentos de una hormiga se relacionan con el resto de la soluci\'on, ya que estos ser\'ia un reflejo indirecto de los movimientos din\'amicos de un filamento en el tiempo, capturados en un punto a trav\'es de una imagen. 
%La rigidez de un filamento puede describirse mediante la relaci\'on que existe entre un segmento de una hormiga, con respecto a todos los segmentos que lo preceden. 
\begin{equation}
    \tau_{ij} \leftarrow
        \begin{cases}
        \begin{split}
         \tau_{ij} \cdot \gamma \text{ si } & \sin(\measuredangle supl(s_{max},s_{min}) > s_{min} \cdot 0.1 \cdot \text{Max\_Axial\_Displacement} \\ & \land \measuredangle supl(s_{max},s_{min}) > \theta    
        \end{split}
        \\[3ex]
        
        \text{0 si } \tau_{ij} \leq 0.25 \\[3ex]
        \tau_{ij} \quad \text{en otro caso}
        \end{cases}
    \label{eq:antiPheroSAP_Axial}
\end{equation}

Si ambas evaluaciones son completadas correctamente y no existen evaluaciones particulares para la c\'elula observada, se declara a la soluci\'on como de buena calidad para un camino que representa un filamento. A diferencia de los filamentos de otras c\'elulas incluidas en este trabajo, el comportamiento esperado de los filamentos en una neurona permite caracterizar los lugares desde los cuales nuevos filamentos pueden generarse. Esta informaci\'on adicional se incorpora mediante la evaluaci\'on particular de un recorrido en una neurona, buscando validar el comportamiento esperado de un filamento, en relaci\'on a que los filamentos de una neurona parten del {\tt soma} o centro de la misma, y que filamentos posteriores que no comiencen del soma s\'olo pueden nacer a partir de otros filamentos. Un camino que termine en un nodo final de grado de 1 no respeta aquel comportamiento, debido a que en un grafo que representa una red de filamentos de una neurona, los nodos ubicados en el soma o en la intersecci\'on entre filamentos presentan un grado mayor a 1. La penalizaci\'on mediante la anti-feromona para este caso particular se refleja en la ecuaci\'on \ref{eq:antiPheroSAP_neuron}.

\begin{equation}
    \tau_{ij} \leftarrow
        \begin{cases}
         \tau_{ij} \cdot \gamma \text{ si } deg(n_{af}) = 1  \\[3ex]
        
        \text{0 si } \tau_{ij} \leq 0.25 \\[3ex]
        \tau_{ij} \quad \text{en otro caso}
        \end{cases}
    \label{eq:antiPheroSAP_neuron}
\end{equation}

%Finalmente, la ecuaci\'on \eqref{eq:antiPheroSAP} refleja la aplicaci\'on de las {\it anti-feromonas} sobre el par $\langle c_{ij}$,$ seg_{n}\rangle \forall c_{ij} \in ]\theta, Max\_Angle]$  donde $c_{ij} \in seg_{n+1}$, y cuya elecci\'on dio lugar a $seg_{n+1}$ que se aleja del desplazamiento axial m\'aximo que un filamento puede soportar.

En relaci\'on a los dominios $D_i$ declarados en la definici\'on del modelo de un COP, se destaca que en el caso de la individualizaci\'on de filamentos existe solo $D_1 \in D$, ya que la instanciaci\'on de variables ($X_i = v_{i}^{j}$ o $c_{ij}$) tiene una sola asignaci\'on posible, lo que lleva a una simplificaci\'on del componente $j$ en las ecuaciones presentadas.

\section{Inicializaci\'on de la metaheur\'istica ACO}

En el paso de inicializaci\'on de ACO se deben definir los valores de los par\'ametros relacionados a las feromonas y las heur\'isticas utilizadas. Para las feromonas se configura el valor inicial de $\tau_{ij}$ en 1 para los pares $\langle c_{ij}$,$ seg_{n}\rangle \> \forall c_{ij} \in \> ]\theta, Max\_Angle]$, dado que al usar SAP esta probabilidad se ir\'a reduciendo de acuerdo a un factor $\gamma$ seg\'un lo explicado en la secci\'on \ref{subsec:pheroUpdate}. El valor de $\gamma$ se define en 0.5, en base a lo encontrado en la literatura.

En el caso de los par\'ametros utilizados en diversas partes del modelo de optimizaci\'on, $\theta$ y {\it Max\_Axial\_Displacement} se encuentran asociados al tipo de c\'elula que se observa, lo que debe ser indicado como informaci\'on a priori. Los valores de $\theta$ son de 30\textdegree ~para microt\'ubulos de planta y de 45\textdegree para neuronas. Por su parte, {\it Max\_Axial\_Displacement} recibe un valor de 1.5 para el caso de los microt\'ublos de planta o de 2.5 para las neuronas. Existen otras opciones de c\'elulas disponibles en la implementaci\'on del modelo de optimizaci\'on. El par\'ametro {\it Max\_Angle} se define como el m\'aximo entre 2.5 veces $\theta$ y 90. El par\'ametro {\it Max\_Score} utilizado por la heur\'istica miope se fija en 2.


Por su parte, el criterio de finalizaci\'on consiste en generar nuevas hormigas hasta que todas las aristas sean parte de al menos una soluci\'on de buena calidad, o que el n\'umero de hormigas generadas sea superior a 4 veces la cantidad de aristas.


En resumen, las condiciones del problema de identificaci\'on de filamentos dan pie a establecer su representaci\'on mediante un problema de optimizaci\'on de restricciones (COP), estando el modelo para la resoluci\'on del COP basado en la metaheur\'istica ACO para su resoluci\'on. La implementaci\'on del modelo de optimizaci\'on presentado es un algoritmo denominado {\it Phil} el que se encuentra escrito en C++.