\chapter{Conclusiones}
%\begin{conclusion}
\label{chap:conclu}
La obtenci\'on de un grafo representativo de una red de filamentos a partir de una imagen es un proceso complejo, cuyo resultado limita la calidad m\'axima del procedimiento de individualizaci\'on de filamentos. Los procedimientos de limpieza sobre la imagen de la c\'elula observada, as\'i como el proceso de esqueletonizaci\'on pueden causar perdida de informaci\'on o introducir deformaciones, impactando en la calidad del grafo que se extrae, y que sirve de base para la individualizaci\'on de filamentos, por lo que resulta imperativo obtener la mayor cantidad de informaci\'on en estos pasos para compensar estos posibles problemas. 

La obtenci\'on de m\'ultiples caracter\'isticas de los filamentos es cr\'itica para descartar rapidamente soluciones de baja calidad, algo especialmente necesario en un problema en el que la combinaci\'on de aristas que conforman un camino que representa un filamento puede crecer exponencialmente. Estas caracter\'isticas pueden provenir no s\'olo de aspectos geom\'etricos de los filamentos, sino que tambi\'en de informaci\'on topol\'ogica y/o espacial de los mismos.
A\'un con aquella informaci\'on que permite explorar de mejor forma las combinaciones de aristas contiguas o caminos que representan un filamento en un grafo, la informaci\'on de la c\'elula observada disponible {\it a priori}, facilita el establecimiento de umbrales y/o cotas que describen parcialmente aspectos del comportamiento de los filamentos.

El an\'alisis derivado de las m\'ultipes caracter\'isticas obtenidas se enfoca en obtener informaci\'on cuantitativa respecto al comportamiento de los filamentos, marcando una diferencia con otras investigaciones con an\'alisis de tipo cualitativo. De forma similar, el algoritmo propuesto cuenta con una estructura flexible que permite incorporar informaci\'on y/o l\'ogicas adicionales, las que dependiendo de la etapa en la metaheur\'istica ACO en que se encuentren, pueden variar la forma en que se explora el espacio de b\'usqueda o en que se aplican restricciones a los resultados parciales obtenidos. Otro aspecto del algoritmo propuesto radica en tener par\'ametros predefinidos dependiendo del tipo de c\'elula, lo que busca facilitar el an\'alisis del experto y a su vez permite automatizar la extracci\'on de informaci\'on para los mismos tipos de c\'elulas.


% se prueba la hipótesis, se obtienen mejores resultados que el estado del arte pero aun falta
Los resultados obtenidos en esta investigaci\'on incorporan los elementos mencionados previamente, permitiendo individualizar filamentos de forma similar o mejor con respecto a m\'etodos en el estado del arte, y a su vez cumpliendo parcialmente con la hip\'otesis de identificar cada filamento resolviendo un modelo de optimizaci\'on. A pesar de no lograr una individualizaci\'on correcta de la totalidad de los filamentos, se obtiene un comportamiento estable, que no presenta diferencias estad\'isticas significativas con las individualizaciones realizadas por los expertos. Cabe volver a destacar lo mencionado en el cap\'itulo \ref{sec:genGrafFromImage} en relaci\'on a que es posible que 2 expertos indiquen diferentes filamentos a partir de la misma imagen, por lo que es necesario incorporar pruebas adicionales a este trabajo para descartar la existencia de un sobre-ajuste del modelo a la informaci\'on provista por los expertos que realizaron las individualizaciones manuales de las im\'agenes utilizadas.


Con respecto al uso de la metaheur\'istica ACO como base del algoritmo propuesto, se propone una diferencia respecto del planteamiento de PCP o FCP, donde una arista debe pertenecer exactamente o al menos a un camino respectivamente, relaj\'andose aquella restricci\'on dado que pueden existir aristas aisladas producto una extracci\'on de grafo con perdida de informaci\'on, ocasionando discontinuidades. La captaci\'on de aquella informaci\'on es relevante dado que existen m\'etodos en el estado del arte que permiten realizar la uni\'on de una arista aislada y el filamento al que potencialmente puede pertenecer, pudiendo incorporarse como etapas adicionales al modelo. 

%cumplimiento de objetivos
En general, el algoritmo propuesto permite resolver un modelo de optimizaci\'on para individualizar filamentos a partir de un grafo con pesos que representa una red de filamentos. En cuanto al cumplimiento de los dem\'as objetivos espec\'ificos, se tiene:
\begin{itemize}
    \item {\bf Implementar un algoritmo que resuelva el modelo de optimizaci\'on, entregando como salida la identificaci\'on de filamentos, considerando casos de solapamiento y/o cruce}: El algoritmo propuesto cumple con este objetivo, de acuerdo a lo presentado en los cap\'itulos \ref{sec:modeloOpti} y \ref{chap:res}.
    
    \item {\bf Identificar la ponderaci\'on de propiedades que entregue mejores resultados para grafos que representen una neurona, una bacteria y una c\'elula eucariota de planta:}: Se realiza una ponderaci\'on de las propiedades utilizadas en el algoritmo propuesto en la secci\'on \ref{subsec:ponderacion}, la que var\'ia dependiendo del tipo de c\'elula y se encuentra asociado a la distribuci\'on de las caracter\'isticas en las distintas etapas de la metaheur\'istica ACO.
    
    \item {\bf Evaluar t\'ecnicas que usan s\'olo herramientas de visi\'on por computador, basadas en poblaci\'on de p\'ixeles y otras que utiliza un m\'etodo derivado de contornos activos:} Las t\'ecnicas indicadas se encuentran descritas en el cap\'itulo \ref{chap:stateoftheart}, y se enfocan en la extracci\'on de informaci\'on geom\'etrica y/o topol\'ogica. Sin embargo no realizan individualizaci\'on de filamentos, por lo que la evaluaci\'on fue dirigida hacia la investigaci\'on que da lugar a DeFiNe, que si realiza este procedimiento, siendo evaluada en el cap\'itulo \ref{chap:res}.
\end{itemize}


Como cotinuaci\'on de esta investigaci\'on se encuentra la opci\'on de intensificar la extracci\'on de informaci\'on topol\'ogica y espacial que permita realizar la asociaci\'on de secciones importantes de los filamentos o elementos alrededor de estos en la c\'elula observada, con el fin de seguir a\~nadiendo informaci\'on al modelo. Tambi\'en es posible vislumbrar una extensi\'on de este trabajo en la identificaci\'on de las interacciones entre los filamentos individualizados. Lo anterior se refleja en microt\'ubulos como la identificaci\'on de interacciones tipo cat\'astrofe, zippering o nucleaci\'on, mientras que para una neurona ser\'ia la clasificaci\'on autom\'atica de los filamentos en axon o dendritas primarias o secundarias.