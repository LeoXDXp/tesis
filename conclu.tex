\chapter{Conclusiones}
%\begin{conclusion}
\label{chap:conclu}
La obtenci\'on de un grafo representativo de una red de filamentos a partir de una imagen es un proceso complejo, cuyo resultado limita la calidad m\'axima del procedimiento de individualizaci\'on de filamentos. Los procedimientos de limpieza de la imagen, as\'i como el proceso de esqueletonizaci\'on pueden causar perdida de informaci\'on o introducir deformaciones, impactando en la calidad del grafo que sirve de base para la individualizaci\'on de filamentos, por lo que resulta imperativo obtener la mayor cantidad de informaci\'on en estos pasos para compensar estos posibles problemas. 

La obtenci\'on de m\'ultiples caracter\'isticas de los filamentos es cr\'itica para descartar rapidamente soluciones de baja calidad, algo especialmente necesario en un problema en el que la combinaci\'on de aristas que conforman un camino que representa un filamento puede crecer exponencialmente.

%cumplimiento de objetivos


A diferencia del planteamiento de PCP o FCP, donde una arista debe pertenecer exactamente o al menos a un camino, respectivamente, en el presente modelo de optimizaci\'on propuesto se levanta aquella restricci\'on dado que pueden existir aristas aisladas producto una extracci\'on de grafo con perdida de informaci\'on asociado a discontinuidades,


Una extensi\'on natural basada en los resultados obtenidos por el modelo de optimizaci\'on presentado en este trabajo es la identificaci\'on de las interacciones entre los filamentos individualizados. Lo anterior se refleja en microt\'ubulos como la identificaci\'on de interacciones tipo cat\'astrofe, zippering o nucleaci\'on, mientras que para una neurona ser\'ia la clasificaci\'on autom\'atica de los filamentos en axon o dendritas primarias o secundarias. 
% \begin{itemize}
%     \item Dificultad para obtener grafo
%     \item uso simultaneo de 2 caracter\'isticas permite obtener mejores resultados
%     \item lo anterior valida la hip\'otesis
%     \item identificaci\'on de casos a partir de interacci\'on de filamentos identificados. Catastrofe, zippering etc para MTs. Axon, filamentos primarios, secundarios etc para Neuronas 
%     \item importancia de la informaci\'on a priori
%     \item se pueden obtener individualizaciones correctas de filamentos en base a grafos que presentan discontinuidades o deformaciones al ser basados en esqueletonizaci\'on.
% \end{itemize}
%\end{conclusion}
