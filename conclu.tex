\begin{conclusion}
\label{chap:conclu}
\begin{itemize}
    \item Dificultad para obtener grafo
    \item uso simultaneo de 2 caracter\'isticas permite obtener mejores resultados
    \item lo anterior valida la hip\'otesis
    \item identificaci\'on de casos a partir de interacci\'on de filamentos identificados. Catastrofe, zippering etc para MTs. Axon, filamentos primarios, secundarios etc para Neuronas 
    \item importancia de la informaci\'on a priori
    \item se pueden obtener individualizaciones correctas de filamentos en base a grafos que presentan discontinuidades o deformaciones al ser basados en esqueletonizaci\'on.3
\end{itemize}
\end{conclusion}
