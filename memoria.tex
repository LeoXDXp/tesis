\documentclass[upright, contnum]{umemoria}

%fix for the oneside argument
\makeatletter
\g@addto@macro\titlepage{\pagenumbering{Alph}}
\g@addto@macro\endtitlepage{\pagenumbering{roman}}
\makeatother

\depto{Departamento de Ciencias de la Computación}
\author{Leonardo Enrique Pizarro Baeza}
\title{Individualizaci\'on de filamentos en una red mediante optimizaci\'on}
\auspicio{Parcialmente financiado por el Fondo Nacional de Desarrollo Cient\'ifico y Tecnol\'ogico (FONDECYT 1180906)}
\date{Noviembre 2019}
\guia{Mauricio Cerda Villablanca \break Jacques Dumais}
\carrera{Magíster en Ciencias, Mención Computación}
\memoria{Tesis para optar al Grado de \break  Magíster en Ciencias, Mención Computación}
\comision{}

\usepackage{lipsum}
\usepackage{natbib}
\usepackage{graphicx}
\usepackage{subcaption}
\usepackage{gensymb}
\usepackage[utf8]{inputenc}
\usepackage[T1]{fontenc}

\begin{document}

\frontmatter
\maketitle

\begin{abstract}{
En la naturaleza, es posible encontrar de forma ubicua, estructuras alargadas (filamentos), las que conforman redes entre s\'i.  La conformaci\'on de estas estructuras complejas y din\'amicas se puede observar en ejemplos
particulares como en una red de prote\'inas de una c\'elula eucariota, as\'i como en bacterias, ya que a pesar que pertenecer a distintas familias, ambas tienen estructuras formada por filamentos. La individualizaci\'on de filamentos permite cuantificar las propiedades de la red tales como n\'umero de filamentos, largo de estos, volumen, o curvatura, y  puede ser categorizado en: basado en procesamiento de im\'agenes o como un problema de optimizaci\'on.

El problema de identificar filamentos en im\'agenes de microscop\'ia est\'a limitado por la resoluci\'on, y los problemas de m\'ultiples par\'ametros a ajustar, para los m\'etodos basados en procesamiento de im\'agenes, el costo
computacional en los m\'etodos basados en optimizaci\'on, y falta de descriptores cuantitativos en ambas. La revisi\'on bibliogr\'afica da cuenta tambi\'en de pocas herramientas disponibles. Todo lo anterior implica que parte del an\'alisis deba ser manual, lo que para grandes cantidades de datos, hace los estudios más propensos a errores.

Esta investigación se centra en el desarrollo un modelo de optimizaci\'on para la individualizaci\'on de filamentos a partir de un grafo representativo de la red de filamentos, evaluando la variaci\'on en el resultado con distintas combinaciones entre las propiedades de cada segmento de filamento como el grosor, largo, \'angulo de bifurcaci\'on o direcci\'on.

Como resultado, se obtiene una herramienta para el an\'alisis automatizado de estructuras de filamentos, permitiendo mejorar el seguimiento de procesos din\'amicos o la identificaci\'on de procesos patol\'ogicos de forma autom\'atica.
}
\end{abstract}

\begin{dedicatoria} % opcional
Dedicado a todos
\end{dedicatoria}

\begin{thanks} % opcional
A todos
\end{thanks}
\cleardoublepage

\tableofcontents
\listoftables % opcional
\listoffigures % opcional

\mainmatter

%\begin{intro}
\chapter{Introducci\'on}
\label{chap:intro}

En la naturaleza, es posible encontrar de forma ubicua, estructuras alargadas (filamentos), las que conforman redes entre sí. La conformación de estas estructuras complejas y din\'amicas se puede observar en ejemplos particulares como en una red de prote\'inas de una c\'elula eucariota, as\'i como en bacterias, ya que a pesar de pertenecer a distintas familias, ambas tienen estructuras formadas por filamentos. 
Caracter\'isticas f\'isicas de estas redes generan propiedades tales como la presencia o ausencia de ciclos, o la posibilidad de dividir o no cada filamento. Por su parte, el análisis de filamentos que conforman la red, puede indicar el estado de \'esta, respecto a su ambiente o de su interior, as\'i como develar informaci\'on relevante sobre la relaci\'on entre la estructura biol\'ogica y funciones fisiol\'ogicas.  
%movimiento interno, reparación de tejido,
% estructuras dinamicas y complejas que juegan varios roles
% A modo de ejemplo, una red de proteinas de una c\'elula eucariota contiene tres tipos de filamentos en la constituci\'on de su citoesqueleto: microfilamentos, microt\'ubulos y filamentos intermedios. Sin embargo, una estructura de citoesqueleto también existe en una bacteria.
  
%\vspace{.5cm}
Los m\'etodos actuales para analizar las redes mencionadas se basan en el procesamiento directo de im\'agenes obtenidas a partir de microscop\'ia (Figura \ref{Fig1a}), pasando por etapas de segmentaci\'on (Figura \ref{Fig1b}), para luego utilizar diversas t\'ecnicas como esqueletonizaci\'on (Figura \ref{Fig1c}), la transformada de Rad\'on o {\it template matching}. Esto permite identificar el grafo que representa a la red (Figura \ref{Fig1d}) o sirve de base para el uso de heur\'isticas que permiten identificar los filamentos de forma individual.
 %como tambi\'en pueden utilizarse una esqueletonizaci\'on de la misma (Figura \ref{Fig1c}), o la construcci\'on de un grafo (Figura \ref{Fig1d}), entre otras.
 % En todos estos casos, el objetivo radica en
La individualizaci\'on de filamentos permite cuantificar las propiedades de la red tales como n\'umero de filamentos, largo, volumen, o curvatura. Estos m\'etodos, basados en la observaci\'on mediante microscop\'ia \'optica tienen como cota m\'axima de resoluci\'on el l\'imite de difracci\'on, $\lambda/2$. Donde $\lambda$ es la longitud de onda de la luz utilizada (o color). Este l\'imite establece que 2 objetos cuya distancia sea inferior a $\lambda/2$ no pueden ser diferenciados, conllevando a que dos partes cercanas del grafo puedan ser observadas como una, dificultando su estudio. Lo anterior es relevante para asociar las propiedades de la red a los v\'ertices del grafo extra\'ido, dando pie a la caracterizaci\'on del mismo (Figura \ref{Fig2a}).
 % vertices del grafo extraido, que representan a un filamento de forma individual, 
 
 \begin{figure*}[h!]
    \centering
    \label{fig:flujo-expected}
    \begin{subfigure}[t]{0.45\textwidth}
        \centering
        \includegraphics[height=1.5in]{imagenes/define-weighted-4.png}
        \caption{Representaci\'on simplificada de una red con cruces y sobreposici\'on de filamentos en una imagen de microscop\'ia.}
        \label{Fig1a}
    \end{subfigure}%
    ~ \hspace{0.5cm}
    \begin{subfigure}[t]{0.45\textwidth}
        \centering
        \includegraphics[height=1.5in]{imagenes/define-weighted-4-bw-invert.png}
        \caption{Preprocesamiento de la imagen mediante segmentaci\'on para la extracci\'on de la red.}
        \label{Fig1b}
    \end{subfigure}
    %\caption{Caption place holder}
%\end{figure*}
\vskip\baselineskip
%\begin{figure*}[t!]
%    \centering
    \begin{subfigure}[t]{0.45\textwidth}
        \centering
        \includegraphics[height=1.5in]{imagenes/skel_in_segment.png}
	    \caption{Esqueletonizaci\'on representativa de la red sobre la imagen de \ref{Fig1b}.}
        \label{Fig1c}
    \end{subfigure}%
    ~ \hspace{0.5cm}
    \begin{subfigure}[t]{0.45\textwidth}
        \centering
        \includegraphics[height=1.5in]{imagenes/graph_of_skel_no_axis.png}
        \caption{Grafo que representa la esqueletonizaci\'on de la red.}
        \label{Fig1d}
    \end{subfigure}
%    \vskip\baselineskip
    \caption{Procedimiento para obtener un grafo que representa la red, a partir del procesamiento de una imagen de microscop\'ia, utilizando segmentaci\'on y esqueletonizaci\'on. Fuente: \cite{breuer2015define}.}
\end{figure*}

Existen investigaciones en la literatura que apuntan a las distintas etapas de este problema, las que tienen en com\'un el uso de t\'ecnicas del \'area de procesamiento de im\'agenes para el tratamiento inicial de la imagen de microscop\'ia. La individualizaci\'on de filamentos puede ser categorizada dependiendo si utiliza como enfoque primario herramientas de procesamiento de im\'agenes, o si se aborda como un problema de optimizaci\'on. 
En ambas categor\'ias, las cr\'iticas m\'as repetidas en los trabajos del \'area suelen ser la cantidad de par\'ametros y la dificultad en su ajuste, en las diversas herramientas existentes. Un segundo problema com\'un es que la obtenci\'on de informaci\'on relacionada a la morfolog\'ia y el comportamiento de las redes es m\'as cualitativo que cuantitativo \cite{asgharzadeh2018computational}\cite{qiu2014quantitative}, lo que supone un problema al trasladar el  an\'alisis a una gran cantidad de datos, dado que cada enfoque es demasiado espec\'ifico a su correspondiente software.

\begin{figure*}[h]
    \begin{subfigure}[t]{0.5\textwidth}
        \centering
        \includegraphics[height=1.2in]{imagenes/define-weighted-4-expected2.png}
        \caption{Visualizaci\'on de un resultado posible de individualizaci\'on, limitando los filamentos identificados a filamentos simples.}
        \label{Fig2a}
    \end{subfigure}
    ~ 
    \begin{subfigure}[t]{0.5\textwidth}
        \centering
        \includegraphics[height=1.2in]{imagenes/define-weighted-4-expected1.png}
        \caption{Visualizaci\'on de otro resultado posible de individualizaci\'on, permitiendo filamentos m\'as complejos, como es el caso de filamentos con ramificaciones. }
        \label{Fig2b}
    \end{subfigure}
	\caption{Posibles resultados de la identificaci\'on de filamentos. Un planteamiento enfocado en un modelo de optimizaci\'on permitir\'a asociar las arista del grafo, como el de la figura \ref{Fig1d}, a uno o m\'as filamentos, entregando resultados como (a) o (b), expresado gr\'aficamente mediante gradiente de colores. Fuente: Elaboraci\'on propia.}
	%El resultado esperado de esta investigaci\'on es asociar a cada arista del grafo (como el de la figura \ref{Fig1d}) un peso que pondere diversos criterios como \'angulo de bifurcaci\'on del filamento, grosor y/o largo.
\end{figure*}

En resumen, el problema de identificar filamentos en im\'agenes de microscop\'ia esta limitado por la resoluci\'on, y los problemas de m\'ultiples par\'ametros a ajustar, para los m\'etodos basados en procesamiento de im\'agenes, el costo computacional en los m\'etodos basados en optimizaci\'on, y falta de descriptores cuantitativos en ambas. La revisi\'on bibliogr\'afica da cuenta tambi\'en de pocas herramientas disponibles. Todo lo anterior implica que parte del an\'alisis deba ser manual, lo que para grandes cantidades de datos, hace los estudios m\'as propensos a errores. 

\section{Hip\'otesis y Objetivos}
Como se ha presentado, los m\'etodos de individualizaci\'on de filamentos que s\'olo usan herramientas de procesamiento de im\'agenes, pueden representar la red a trav\'es de los filamentos que la conforman, sin embargo para realizar mediciones cuantitativas requieren ajustar m\'ultiples par\'ametros. Por otra parte los m\'etodos de optimizaci\'on permiten reducir los par\'ametros, pero con un alto costo computacional, el que puede ser abordado con el uso de heur\'isticas o algoritmos de aproximaci\'on. Tambi\'en se debe tener en consideraci\'on en ambos m\'etodos, el n\'umero de caracter\'isticas utilizadas al momento de realizar la individualizaci\'on. Esto nos permite establecer la pregunta:

\smallskip
¿Es posible determinar correctamente la cantidad de filamentos, desarrollando un modelo de optimizaci\'on que utilice una combinaci\'on de m\'ultiples caracter\'isticas como el largo, grosor, \'angulo de bifurcaci\'on, curvatura o direcci\'on, tanto en la asignaci\'on de pesos como en la selecci\'on del subconjunto de caminos?
\smallskip

% costo computacional de m\'etodos de optimizaci\'on es abordado con algoritmos de aproximación o heurísticas
% Solucion actual propuesta en Define aborda el recorrido de grafos considerando como peso solo 1 valor que representa una caracteristica de la arista, 
% pregunta es si se puede determinar la cantidad de filamentos utilizando un peso que agrupe/pondere varias caracteristicas
%(geometricos comunmente)
% Entre las investigaciones y herramientas existentes derivadas de los mismos, 
%- overlap, intensidad, fragmentación

Dado que en el problema de individualizaci\'on de filamentos se desconoce el origen y fin de cada filamento al iniciar la resoluci\'on del problema, plantear un recorrido de grafos puede resultar computacionalmente costoso, lo que puede ser evitado al plantear un modelo de optimizaci\'on que pueda hacer la asociaci\'on entre aristas que sean parte del mismo filamento.

Se debe agregar que existe informaci\'on \textit{a priori}, indicada por el tipo de estructura observada, la cual restringe los comportamientos posibles de la red. Por ejemplo, los filamentos de prote\'ina actina no se pegan entre s\'i, formando estructuras sin ciclos, mientras que el ret\'iculo endoplasm\'atico (organelo celular encargado principalmente de la s\'intesis de prote\'inas) si presenta ciclos. Estas condiciones pueden aportar limites m\'as acotados a algunos de los criterios de cuantificaci\'on y es conocimiento disponible previo a la observaci\'on en el microscopio. Es importante destacar que alguna de la informaci\'on {\it a priori} corresponde a reglas emp\'iricas, que tambi\'en deben ser consideradas durante el an\'alisis.

En base a lo anterior, la hip\'otesis de esta investigaci\'on se basa en que a partir de un grafo con pesos, no dirigido, con o sin ciclos, que representa una red de filamentos, en conjunto con utilizar una combinaci\'on de caracter\'isticas de los segmentos de filamentos como largo, grosor, \'angulo de bifurcaci\'on o curvatura, sumado a la incorporaci\'on de informaci\'on previa disponible sobre el tipo de c\'elula (red de filamentos con o sin ciclos), es posible identificar a cada filamento en la red resolviendo un modelo de optimizaci\'on.

Para validar esta hip\'otesis, se propone realizar los siguientes objetivos:
\subsection{Objetivo General}
Desarrollar un modelo de optimizaci\'on para la individualizaci\'on de filamentos a partir de un grafo representativo de la red de filamentos, evaluando la variaci\'on en el resultado con distintas combinaciones entre las propiedades de cada segmento de filamento como el grosor, largo, \'angulo de bifurcaci\'on o direcci\'on.
%y que incorpore la informaci\'on {\it a priori} disponible

\subsection{Objetivos Espec\'ificos}
\begin{itemize}
    \item Generar un modelo de optimizaci\'on para la identificaci\'on de filamentos a partir de un grafo con pesos que representa la red de filamentos.
    \item Implementar un algoritmo que resuelva el modelo de optimizaci\'on, entregando como salida la identificaci\'on de filamentos, considerando casos de solapamiento y/o cruce.
    \item Identificar la ponderaci\'on de propiedades que entregue mejores resultados para grafos que representen una neurona, una bacteria y una c\'elula eucariota de planta.
    \item Evaluar t\'ecnicas que usan s\'olo herramientas de visi\'on por computador \cite{boudaoud2014fibriltool}, basada en poblaci\'on de p\'ixeles, o que utilizan un m\'etodo derivado de contornos activos \cite{xu2015soax}.
    %una soluci\'on exacta o aproximada respecto a la individualizaci\'on, dependiendo de la complejidad computacional del problema. 
\end{itemize}

%\subsubsection{Contribuci\'on}

% Contribuci\'on:
Como resultado de los objetivos planteados, se obtendr\'a una herramienta para el an\'alisis automatizado de estructuras de filamentos, lo que permitir\'ia mejorar el seguimiento de procesos din\'amicos o la identificaci\'on de procesos patol\'ogicos de forma autom\'atica.

\section{Estructura de esta tesis}
El capitulo \ref{chap:stateoftheart} resume el estado del arte de diversos aspectos relacionados a la individualizaci\'on de filamentos. El cap\'itulo \ref{sec:modeloOpti} define un modelo de optimizaci\'on que resuelve la identificaci\'on de filamentos, para el que se propone un algoritmo. Se presentan las m\'etricas y mediciones en el cap\'itulo \ref{chap:metodologia}. El cap\'itulo \ref{chap:res} presenta los resultados obtenidos para im\'agenes sint\'eticas e im\'agenes reales de distintas c\'elulas. Finalmente, en el cap\'itulo \ref{chap:conclu} se presentan las conclusiones.

%\begin{teo}
%Se tiene que $$\int_0^t e^sds=e^t-1.$$
%\end{teo}
%\end{intro}
%\chapter{Estado del Arte}
\label{chap:stateoftheart}
De la recopilaci\'on de antecedentes encontrados en la literatura, es posible separar los m\'etodos en dos categor\'ias. En la primera categor\'ia se encuentran enfoques asociados al an\'alisis de la reconstrucci\'on de filamentos en base a sus segmentos, como \cite{zhang2017extracting}, que identifica el problema de discontinuidad de filamentos en una imagen, pudiendo atribuirse esto a factores experimentales como la densidad del componente fluorescente, ruido u otras. Para llevar a cabo la reconstrucci\'on de un filamento en base a segmentos, establecen el uso de dos filtros, denominados \textit{Filtro de Transformaci\'on Lineal} (LFT por su sigla en ingl\'es) y \textit{Filtro de Transformaci\'on en base a la Orientaci\'on} (OFT por su sigla en ingl\'es). 
El filtro LFT busca resaltar caracter\'isticas lineales centr\'andose en cada p\'ixel y generando una serie de lineas con radio r, buscando la linea que contenga la mayor intensidad, seg\'un se observa en la figura \ref{fig:MTLFT}. El filtro OFT complementa lo anterior, mediante un criterio de similaridad en la direcci\'on proyectada de los segmentos, un segundo criterio respecto a la distancia entre los extremos finales de cada segmentos y un tercer criterio de continuidad, que limita el \'angulo entre vector proyectado de un extremo y el vector que representa la distancia entre los extremos de cada segmento. Lo anterior se puede observar en la figura \ref{fig:MTOFT}.

\begin{figure}[h]
        \centering
        \includegraphics[scale=5]{imagenes/MT-LFT.jpeg}
        \caption{B\'usqueda por lineas con radio r para diferentes \'angulos, según filtro LFT. Fuente: \cite{zhang2017extracting}}
        \label{fig:MTLFT}
\end{figure}

\begin{figure}[h]
        \centering
        \includegraphics[scale=1.3]{imagenes/MT-OFT.jpeg}
        \caption{Criterios del filtro OFT: (A) Vectores de propagaci\'on (rojo) y distancia (amarillo) (B) Ejemplos de criterios de coincidencia de fragmentos de filamento I) Similaridad de direcci\'on, II) Proximidad, III) Continuidad. Fuente: \cite{zhang2017extracting}}
        \label{fig:MTOFT}
\end{figure}


El fundamento de los criterios de OFT se basa en la asociaci\'on de las caracter\'isticas geom\'etricas como restricciones representativas del comportamiento mec\'anico de los microt\'ubulos. Uno de los objetivos de esta investigaci\'on es permitir el an\'alisis autom\'atico, dada las dificultades que el an\'alisis manual de reconstrucci\'on presenta. % es dificil hacer el analisis de forma manual
%asume non-branching filaments entonces soluciona el caso


En esta misma categor\'ia es posible clasificar a las investigaciones de extracci\'on de la red que representa a los filamentos a partir de una imagen\cite{doi:10.1021/ma502264c}\cite{boudaoud2014fibriltool}\cite{lichtenstein2003quantitative}\cite{alioscha2016robust}\cite{xu2015soax}\cite{asgharzadeh2018computational}.
% hablar de fibritool, Quantitave IFS y SOAX
%fibritool
De este subgrupo, \cite{boudaoud2014fibriltool} desarrolla \textit{FibriTool}, un plugin para el software ImageJ, utilizado en an\'alisis cient\'ifico de im\'agenes. FibriTool calcula la orientaci\'on principal y la anistrop\'ia de estructuras alargadas dentro de una regi\'on de inter\'es seleccionada manualmente. Para esto, utiliza el concepto de tensor nem\'atico, extra\'ido del comportamiento f\'isico de los cristales l\'iquidos. Espec\'ificamente, el tensor nem\'atico es la matriz sim\'etrica $n$ de $2\times2$ construida a partir de un vector unitario $t$, ecuaci\'on \eqref{eq:fibritoolTensor}, definido en base a la derivada de primer orden de la intensidad del pixel en $x,y$. Los componentes de $n$ se pueden observar en la ecuaci\'on \eqref{eq:fibritoolComps}.


\begin{subequations}
\label{eq:fibritoolComps}
Componentes de $n$ para el tensor nem\'atico:
\begin{align}
    n_x,x &= (t_x)^2 \\
    n_y,y &= (t_y)^2 \\
    n_x,y &= n_y,x = t_x \times t_y
\end{align}
\end{subequations}


Luego, a partir de la matriz $n$ se obtiene su primer eigenvector $e_1$ que representa la orientaci\'on principal de los filamentos en el \'area de inter\'es, mientras que la diferencia de los eigenvalores $n_1$ y $n_2$ ($n_1 > n_2$), denominada $q = n_1 - n_2$, define la anisotrop\'ia.

\begin{equation}
\label{eq:fibritoolTensor}
t = (t_x,t_y) = (
\dfrac{\partial I}{\partial y}, -\frac{\partial I}{\partial x}) / \sqrt{  
(\frac{\partial I}{\partial x})^2 + 
(\frac{\partial I}{\partial y})^2 }
\end{equation}
 
Los autores de FibriTool desarrollan este m\'etodo para evitar el uso de derivadas de segundo orden, que presentan sensibilidad al ruido, necesitando de pasos previos en la limpieza de la imagen.

\begin{figure}[h!]
        \centering
        \includegraphics[scale=0.6]{imagenes/fibritool.jpg}
        \caption{\'Area de inter\'es seleccionada manualmente en amarillo, y resultado de FibriTool en rojo o verde. Fuente: \cite{boudaoud2014fibriltool}}
        \label{fig:fibritool}
\end{figure}

% Quantitative IFS
Otra investigaci\'on de extracci\'on e individualizaci\'on de filamentos es \cite{qiu2014quantitative}, la que propone un filtro de detecci\'on de caracter\'isticas de 6 pasos m\'as un algoritmo llamado \textit{SBDA} que busca eliminar segmentos de filamentos menores a 2 p\'ixeles (Figura \ref{fig:IFS}). 

%describir filtros, intensidad es con fibre coefficient
Las 6 etapas del filtro consideradas en este m\'etodo consisten en 4 etapas de an\'alisis de la imagen y 2 de an\'alisis topol\'ogico que se resumen en lo siguiente:

\begin{enumerate}
    \item El primer filtro comienza por reducir el ruido y realzar el contraste de la imagen, aumentando la intensidad de los p\'ixeles que est\'en sobre un umbral, al mismo tiempo que eliminan los p\'ixeles que se encuentren debajo del mismo umbral.
    \item El segundo paso consiste en filtrar informaci\'on estructural, que se basa en los eigenvalores de la matriz Hessiana, buscando descartar objetos en la imagen que no correspondan a una figura tubular.
    \item El tercer filtro ejecuta una limpieza sobre lo que se define como \textit{se\~nales d\'ebiles}, pudiendo ser estructuras con una intensidad baja y/o aisladas y por ende no corresponden a elementos en el plano focal de inter\'es.
    \item Finalmente dentro del an\'alisis sobre la imagen, el filtro de esqueletonizaci\'on realiza un adelgazamiento de la imagen, en el que cada estructura pasa a tener 1 p\'ixel de ancho, facilitando el an\'alisis topol\'ogico posterior.
    \item La clasificaci\'on topol\'ogica consiste en el an\'alisis a nivel de p\'ixel y su vecindario de 8 p\'ixeles alrededor para determinar si este corresponde a un punto aislado, al final de un fragmento de filamento, a un punto interior de un filamento, o a una junci\'on de filamentos. Seguido de aquello, el algoritmo SBDA realiza otro an\'alisis topol\'ogico a nivel de p\'ixel que borra los segmentos menores a 3 p\'ixeles, adem\'as de realizar los calculos de distancia de cada segmento/fragmento.
    \item Para finalizar el an\'alisis topol\'ogico y el proceso de filtrado, se lleva a cabo una combinaci\'on de segmentos para construir los filamentos basandose en el par\'ametro $W$, llamado \textit{ancho efectivo}, cuyo valor define si la uni\'on de 2 o m\'as segmentos se trata de una bifurcaci\'on, una intersecci\'on o un sobrelapamiento. 
\end{enumerate}

\begin{figure}[H]
        \centering
        \includegraphics[scale=0.75]{imagenes/QuantitativeIFS.png}
        \caption{Etapas de individualizaci\'on de filamentos de \cite{qiu2014quantitative}. (A): Imagen de red de fibras de un osteoblasto .(B, C, D, E, F): Filtros de limpieza, tubularidad, segmentaci\'on, conectividad y esqueletonizaci\'on. (G): Clasificaci\'on topol\'ogica de intersecciones a nivel de pixel. (H): Resultado de algoritmo SBDA. (I): Individualizaci\'on de segmentos de filamentos seg\'un forma estructural: aislado (verde), solapado (morado) u otro (azul). Fuente: \cite{qiu2014quantitative}}
        \label{fig:IFS}
\end{figure}
%describir SBDA, el analisis es a nivel de pixel

% se mide caracteristicas geometricas, como largo, distribucu\'on de la orientaci\'on, y como los resultados son afectados por el ruido.
Como resultados de \cite{qiu2014quantitative}, se obtienen caracter\'isticas geom\'etricas como el largo de los filamentos y la distribuci\'on de la orientaci\'on de los filamentos, as\'i como el cambio de estos valores para variaciones en la relaci\'on entre la se\~nal y el ruido de la imagen.

\smallskip
%SOAX, extraccion y cuantifiacion de la red
La investigaci\'on de \cite{xu2015soax}, llamada SOAX, utiliza curvas param\'etricas de contorno abierto (SOAC en ingl\'es) en conjunto con una funci\'on de minimizaci\'on para obtener un peque\~no set de candidatos de soluciones \'optimas, de las que el usuario puede elegir una. Las curvas param\'etricas de contorno abierto son reguladas por 2 par\'ametros: $\tau$, que fija el umbral de intensidad desde el que se inicializa una SOAC, lo que entrega los puntos de intensidad m\'aximos locales de la imagen para la inicializaci\'on. El segundo par\'ametro, $K_{str}$, es el factor que regula la elongaci\'on/evoluci\'on de cada SOAC. Una vez concluida la inicializacio\'on y evoluci\'on de curvas SOAC, se identifican las junciones/intersecciones generadas entre diversas curvas SOAC, agrup\'andose seg\'un su cercan\'ia. Lo anterior se puede observar en la figura \ref{fig:SOAX}.

\begin{figure}[h]
        \centering
        \includegraphics[scale=0.75]{imagenes/SOAX.jpg}
        \caption{Etapas de SOAX: A partir de los puntos de alta intensidad se inicializan curvas param\'etricas de contorno abierto (SOAC en ingl\'es), las que crecen, generando intersecciones entre ellas. Finalmente, se agrupan las intersecciones más cercanas. Fuente: \cite{xu2015soax}}
        \label{fig:SOAX}
\end{figure}

La funci\'on de minimizaci\'on que los autores denominan como  \textit{F-Function}, ecuaci\'on \eqref{eq:FFunction}, se encuentra definida por otros 2 par\'ametros: El factor $c$ ($c > 1$) que regula la penalizaci\'on de las SOACs con bajo \textit{Signal to Noise Ratio (SNR)}, y el umbral $t$, que define el valor mínimo de SNR necesario para no descartar una SOAC. El umbral $t$ se ve reflejado en $L_{<t}$ dentro de la \textit{F-Function}, como el largo de SOACs en una regi\'on de la imagen con SNR por debajo de $t$, y que ser\'an penalizados dado que son calificados como de baja certeza. $L_{total}$ representa el largo total de las SOACs en el resultado final.

\begin{equation}
   \label{eq:FFunction}
    F = -L_{total} + {c}\cdot L_{<t} 
\end{equation}


Para estos m\'etodos, se han indicado como cr\'iticas la dificultad para identificar correctamente un filamento de otro, en los casos de  superposici\'on, fragmentaci\'on, o variaciones de intensidad en la imagen.


\smallskip
La investigaci\'on de \cite{alioscha2016robust} presenta un marco para el an\'alisis de im\'agenes con la finalidad de identificar filamentos de actina, mediante una combinaci\'on de filtros con un algoritmo de uni\'on de segmentos. La propuesta inicial se basa en que una imagen puede ser separada en 3 componentes: el fondo o {\it background} de la imagen, los filamentos y el ruido. Para obtener la imagen que contiene los filamentos y descarta en mayor medida el ruido, los autores utilizan la libreria MCALab en MATLAB. Luego, buscan intensificar los pixels que corresponden a los filamentos, para lo cual aplican 3 filtros: un filtro Gaussiano, un filtro Laplaciano y un filtro Gaussiano direccional, obteniendo como resultado lo que se muestra en la figura \ref{fig:AlioshaRobust}.

\begin{figure}[t]
    \centering
    \includegraphics[scale=2]{imagenes/Aliosha2016-GaussLaplFilters.jpg}
    \caption{A partir de la segmentaci\'on obtenida con la libreria MCALab (imagen a), se aplican los filtros Gaussiano, Laplaciano y Gaussiano direccional para obtener una intensificaci\'on de los filemntos (imagen b). Fuente: \cite{alioscha2016robust}}
    \label{fig:AlioshaRobust}
\end{figure}

El paso siguiente es la aplicaci\'on de un detector de l\'ineas multi-escala, que busca determinar la pertenencia de un p\'ixel a una l\'inea, que varia su grosor y \'angulo dependiendo del tama\~no/escala del vecindario elegido. Las l\'ineas var\'ian en su grosor $s \in [1,W]$, con $W$ el largo esperado de una fibra de actina, as\'i como en su \'angulo, que varia de forma discreta entre 0 y 180\textdegree. El detector de l\'ineas asigna una puntuac\'on que determina la probabilidad del p\'ixel a pertenecer a una de las l\'ineas. Lo anterior entrega como resultado candidatos de segmentos de l\'ineas, con cada una representada por una ecuaci\'on param\'etrica ($\Digamma, \theta$). Para elegir a un candidato como segmento de l\'inea definitivo, la idea principal es recorrer los p\'ixeles en secuencia y ajustar las l\'ineas candidatas usando el m\'etodo de m\'inimos cuadrados hasta que un umbral de error es sobrepasado. Por cada vez que se sobrepasa el umbral de error, se comienza la elecci\'on de un nuevo segmento de l\'inea para los p\'ixeles que siguen. Si el candidato de segmento de l\'inea tiene un largo $l < L$, se descarta. $L$ es un par\'ametro definido experimentalmente.
% esto arroja candidatos de segmentos de lineas -> elegir segmentos de l\'ineas
% despues hay merge de segmentos lineas -> segmentos de filamentos, fibras en este caso

Finalmente, para construir las fibras de actina, los segmentos de l\'inea elegidos en el paso anterior son unidos entre si mediante una ventana de largo $L$, en la que 2 segmentos de l\'inea se superpongan y no tengan una diferencia entre sus \'angulos mayor al umbral $T_{\theta}$. 
%\cite{asgharzadeh2018computational}



\smallskip
En la segunda categor\'ia, \cite{cerda2014geometrical} plantea la identificaci\'on de segmentos de filamentos como un problema de asignaci\'on, utilizando las medidas de distancia euclidiana y angular como restricciones, y el algoritmo h\'ungaro para su resoluci\'on. En la misma categor\'ia, \cite{breuer2015define} realiza la asociaci\'on de la red con un grafo no dirigido con pesos, en el que cada filamento es equivalente a un camino en el grafo. Esto permite que la b\'usqueda e individualizaci\'on de filamentos, con un segmento de filamento representado por una arista del grafo, sumado a las restricciones que plantea el autor, sea tratado como un problema de {\it Set Cover}. En el caso de los m\'etodos de la segunda categor\'ia, la mayor cr\'itica es su costo computacional alto, lo que limita en parte aquel enfoque. Se debe agregar que los par\'ametros utilizados por estas t\'ecnicas (\'angulos o  distancias m\'aximas entre filamentos) son complejas de obtener de los expertos directamente. Sin embargo, una de sus ventajas es que automatizan la recuperaci\'on de informaci\'on incluyendo una mayor cantidad de propiedades a cada arista. 


%En particular, el m\'etodo {\it FCP} presentado en la secci\'on \ref{Antecedentes}, utiliza s\'olo el grosor como caracter\'istica en la etapa de asignaci\'on de pesos a las aristas del grafo, lo que es usado en la funci\'on de minimizaci\'on del problema de optimizaci\'on planteado por ellos. En la etapa de selecci\'on del subconjunto de caminos, el mismo m\'etodo desarrolla una heur\'istica que utiliza {\it BFS} en conjunto con el \'angulo de deflexi\'on entre aristas. El mismo m\'etodo evita realizar un recorrido de grafos, al buscar las combinaciones de caminos del subconjunto P' que dan lugar a uno o m\'as filamentos mediante la minimizaci\'on de un {\it Set Cover}. 

%representa un camino en este grafo, y los conjuntos de caminos 
\medskip
%Todos estos enfoques, cuya entrada principal de datos son las im\'agenes etiquetadas a trav\'es de marcadores fluorescentes, 
% basados en grafos

A modo de ejemplo, el programa \texttt{DeFine} desarrollado en \cite{breuer2015define} describe el problema de identificaci\'on de filamentos como un problema de b\'usqueda de conjuntos de aristas en un grafo. En esta investigaci\'on, un filamento es representado por un conjunto de aristas, y es denominado como un camino. Los autores de esta investigaci\'on se basan inicialmente en un problema del tipo {\it Path Cover}, extendiendo la definici\'on mediante la asignaci\'on de un peso a cada segmento/arista del grafo, para calcular la {\it aspereza} o diferencia de homogeneidad en un camino. El c\'alculo del peso a lo largo de un camino es lo que permite individualizar filamentos, siendo el peso un reflejo del grosor o intensidad de una arista.
%, o tambi\'en puede ser calculado respecto al \'angulo entre aristas.
Este problema particular es llamado {\it Filament Covering Problem} (FCP) y los autores demuestran que es NP-Hard, por lo que proponen un algoritmo de aproximaci\'on mediante \textit{Set Cover}, cuyo objetivo es que cada arista pertenezca al menos a un camino (conjunto de aristas). La definici\'on de {\it Set Cover} es:

\begin{quote}
Dado un conjunto de elementos, denominado universo, y $n$ conjuntos cuya unión comprende el universo, el {\it Set Cover Problem} consiste en identificar el menor n\'umero de conjuntos cuya unión a\'un contiene todos los elementos del universo.
\end{quote}

La adaptaci\'on a lo definido por {\it FCP} es: 
\begin{quote}
Sea el universo $U$ conformado por las aristas del grafo, y un conjunto $S$, conformado por conjuntos de aristas, cada uno con costo $c_s$, $s \in S$:

Encontrar un subconjunto $S_{set} \subseteq S$ con costo m\'inimo (o promedio, dependiendo de la forma en que se calcula la aspereza) tal que cada elemento en $U$ este cubierto al menos una vez.
\end{quote}

Con la demostraci\'on de que FCP es NP-Hard, dado que la cantidad de caminos posibles crece de forma exponencial respecto al n\'umero de nodos, los autores de \cite{breuer2015define} plantean la elecci\'on de un subconjunto de caminos para ser analizados como datos de entrada en un problema de {\it Set Cover}. Este an\'alisis se realiza a trav\'es de un algoritmo de aproximaci\'on lineal fraccional binario. El subconjunto de caminos puede obtenerse con la heur\'istica de recorrer el grafo por su anchura ({\it Breadth-First Search}), deteni\'endose al momento que el \'angulo de deflexi\'on entre aristas adyacentes supere los 60\degree, o de caminos generados a partir de 100 \'arboles de expansi\'on m\'inima aleatoria ({\it RMST}), aportando cada uno con $N(N-1)/2$ caminos no triviales y sin direcci\'on. El flujo de decisiones para generar el subconjunto de caminos puede observarse en la figura \ref{fig:define-set-cover}.

\begin{figure*}[h]
    \centering
    \label{fig:flujo-expected}
    \begin{subfigure}[t]{\textwidth}
        \centering
        \includegraphics[scale=0.5]{imagenes/flujoDefine.png}
        \caption{Entrada de datos y elecci\'on de subconjunto de caminos {\bf P'} en DeFine.}
        \label{fig:define-set-cover}
    \end{subfigure}%
    \vskip\baselineskip
    \begin{subfigure}[t]{\textwidth}
        \centering
        \includegraphics[scale=0.8]{imagenes/BFSvsRMSTpaths.png}
        \caption{Subconjunto {\bf P'} para el grafo de 5 nodos a la izquierda, utilizando la opci\'on de 100 {\it RMST} o heur\'istica de {\it BFS} que corta el camino al encontrar un \'angulo de deflexi\'on mayor a 60\degree entre aristas adyacentes.  }
        \label{fig:subconjunto-p-prima-caminos-posibles}
    \end{subfigure}
    \caption{(\ref{fig:define-set-cover}) A partir de un grafo ponderado proveniente de una estructura sint\'etica o de una imagen real, se elige {\bf P'} entre los $N(N-1)/2$ caminos no triviales y sin direcci\'on que aporta cada uno de los 100 \'arboles de expansi\'on m\'inima aleatoria ({\it RMST}), o los caminos resultantes de la heur\'istica de b\'usqueda por anchura ({\it BFS}) con interrupci\'on al dar con una arista que tenga un \'angulo superior a 60\degree. (\ref{fig:subconjunto-p-prima-caminos-posibles}) Subconjunto {\bf P'} de P, para el grafo de 5 nodos de ejemplo a la izquierda. }
    \end{figure*}
    
    
Este enfoque faculta que al tener un grafo que representa la red de filamentos, como en la figura \ref{Fig1d}, sea posible llegar a resultados como los que aparecen en las figuras \ref{Fig2a} o \ref{Fig2b} a trav\'es de la asignaci\'on de pesos a la aristas, y de restricciones a las uniones entre las mismas. 


Cabe destacar que el {\it FCP} s\'olo utiliza 2 caracter\'isticas independientes para describir los segmentos de los filamentos, siendo el \'angulo de deflexi\'on entre aristas usado en la etapa de selecci\'on de subconjuntos de caminos s\'olo si es seleccionada la heur\'istica {\it BFS}, y el grosor o intensidad, empleado para describir el peso de las aristas. En el caso de la heur\'istica de {\it RMST}, se asignan pesos aleatorios uniformemente distribuidos a las aristas, por lo que no hay uso de caracter\'isticas asociadas a los filamentos.


%resolucion, cantidad parametros, descriptores, costo computacional
En resumen, el problema de identificar filamentos en im\'agenes de microscop\'ia esta limitado por la resoluci\'on, y los problemas de m\'ultiples par\'ametros a ajustar, para los m\'etodos basados en procesamiento de im\'agenes, el costo computacional en los m\'etodos basados en optimizaci\'on, y falta de descriptores cuantitativos en ambas. La revisi\'on bibliogr\'afica da cuenta tambi\'en de pocas herramientas disponibles. Todo lo anterior implica que parte del an\'alisis deba ser manual, lo que para grandes cantidades de datos, hace los estudios m\'as propensos a errores. 
%La falta de herramientas analíticas para cuantificar las estructuras sigue siendo un cuello de botella, ya que el análisis manual de grandes conjuntos de datos requiere una gran cantidad de tiempo y son propensos a sesgos y errores.
%Por otra parte, la soluci\'on a problemas de superposici\'on se soluciona mediante la sintonizaci\'on de par\'ametros, los que var\'ian dependiendo de la c\'elula observada. En el caso del uso de \textit{thinning} en una imagen, la informaci\'on respecto al grosor de la estructura se pierde.

%es de complejidad $\mathcal{O}(N^{(2K+2)})$, con $N$ como n\'umero de nodos y $k$ como el n\'umero m\'aximo de caminos que pasan por cada v\'ertice. 
% Aquello permitir\'ia pasar de Fig1d a Fig2b por ejemplo

%\begin{defn}[ver \cite{KAR00}] Definición definitiva %$$\frac{d}{dx}\int_a^xf(y)dy=f(x).$$\end{defn}
%%\chapter{Problem\'atica de la Individualizaci\'on de Filamentos a partir de un Grafo}
\chapter{An\'alisis de Filamentos en Grafos}
\label{chap:cap2}

%problema previo, o problema 0 que consiste en la generaci\'on de un grafo a partir de una imagen que contenga una red, que en este caso, representaria a una red de filamentos. 

En base lo expuesto en el cap\'itulo \ref{chap:stateoftheart}, para el enfoque de utilizar grafos para la individualizaci\'on de filamentos existe una brecha entre la obtenci\'on del grafo y su an\'alisis. Este paso previo, la extracci\'on de un grafo a partir de una imagen, consiste en extraer un grafo $G = (V,E)$ de una imagen tal que $G$ sea un grafo simple, no dirigido, ponderado, conectado o desconectado, con o sin ciclos. Esto implica que exista a lo m\'as 1 arista por cada par de nodos adyacentes, prohibi\'endose la existencia de nodos conectados consigo mismos. Se definen los v\'ertices/nodos del grafo $G$ como $V(G)$ y las aristas de $G$ como $E(G)$. 
Que el grafo $G$ sea ponderado implica que para las aristas del grafo ($E(G)$), existen caracter\'isticas asociadas que se expresan como caracter\'isticas geom\'etricas, topol\'ogicas, espaciales y/u otras. Es importante evitar que $G$ sea un grafo completo, dado que con n nodos/v\'ertices $G$ llega a tener $\frac{n(n-1)}{2}$ aristas.
%$\forall e \in \quad E(G) \quad  \exists $ 

Algunas de las dificultades involucradas en la extracci\'on de informaci\'on a partir de una imagen se encuentran en los m\'etodos presentados en el cap\'itulo \ref{chap:stateoftheart}, dentro de las que destacan el ruido y la resoluci\'on. Un ejemplo de aquello se observa en la figura \ref{fig:NoConsenso}.

\begin{figure*}[h]
    \begin{tabular}{c c c}
        \multirow[c]{2}{*}[2.5cm]{
        \begin{subfigure}[t]{0.4\textwidth}
        \includegraphics[scale=0.5]{imagenes/NoConsenso.png}
        \caption{Microt\'ubulos en planta {\it Marchantia}.\\Fuente: Paula Llanos}
        \label{fig:NoConsensoGeneral}
        \end{subfigure}  
        }
        &
        \multirow[c]{2}{*}[2cm]{
        \begin{subfigure}[t]{0.25\textwidth}
        \includegraphics[]{imagenes/NoConsenso2.png}
        \caption{Secci\'on resaltada en rojo de \ref{fig:NoConsensoGeneral}}
        \label{fig:NoConsensoRect}
        \end{subfigure}
        }
        &
        \begin{subfigure}[t]{0.21\textwidth}
        \includegraphics[scale=0.8]{imagenes/NoConsenso3.png}
        \caption{Opci\'on 1 de microt\'ubulos en \ref{fig:NoConsensoRect}}
        \label{fig:NoConsensoOpcion1}
        \end{subfigure} \\
        & &
        \begin{subfigure}[b]{0.21\textwidth}
        \includegraphics[scale=0.8]{imagenes/NoConsenso4.png}
        \caption{Opci\'on 2 de microt\'ubulos en \ref{fig:NoConsensoRect}}
        \label{fig:NoConsensoOpcion2}
        \end{subfigure} \\
    \end{tabular}
    
    \caption{Dificultad de individualizaci\'on que enfretan los expertos al analizar manualmente una imagen de filamentos, en particular, microt\'ubulos.}
    \label{fig:NoConsenso}
\end{figure*}

Mientras que el ruido ha sido estudiado en la literatura, el problema de resoluci\'on de filamentos depende principalmente de la capacidad del microscopio que se utilice. Como se menciona en la introducci\'on, el l\'imite m\'aximo de resoluci\'on, denominado $\frac{\lambda}{2}$, determina el tama\~no m\'inimo que 2 objetos que se encuentren juntos pueden tener para no observarse como un \'unico elemento. Lo anterior sucede para algunos tipos de filamentos como los microt\'ubulos que pueden medir tan solo 25 nan\'ometros, lo que se encuentra por debajo de $\frac{\lambda}{2}$ para diversos microscopios. 



Una vez obtenido el grafo que representa una red de filamentos, el problema siguiente se encuentra en el que 2 expertos pueden discernir con respecto a los filamentos identificables en una imagen, por lo que no es posible conocer a priori del origen y el final de un filamento, para las resoluciones actuales de las imagenes obtenidas a partir de la microscop\'ia. Una dificultad adicional se presenta en la representaci\'on de un filamento en un grafo, ya que esta se basa en un conjunto de aristas adyacentes, denominadas caminos o recorridos, lo que lleva a tener un universo de hasta $n!$ posibles combinaciones en el espacio de soluciones en el que se busca un camino.

A partir del problema anterior, el problema final en la individualizaci\'on de filamentos lo constituye la elecci\'on del subconjunto de caminos, que debe ser seleccionado entre el total de caminos que representan soluciones factibles. Esto implica que el problema no solo sea un problema combinatorial de generar soluciones factibles a partir del conjunto de aristas, sino que adem\'as debe considerar la discriminaci\'on entre estos para obtener el subconjunto de mayor calidad, pudiendo representarse como un problema de optimizaci\'on combinatorial.

Los 3 problemas presentados se formalizan a continuaci\'on.

%problema previo
%presentar el problema previo, como un puente necesario en la automatizaci\'on de la extracci\'on,  para analizar un grafo que representa la red de filamentos, que de lo contrario tendria que ser realizado a mano, implicando que la persona realizando el análisis podría llevar a cabo la individualizaci\'on de filamentos en el mismo acto.

\section{Generaci\'on de un Grafo desde una Imagen}
La extraci\'on o generaci\'on de un grafo que representa una red de filamentos a partir de una imagen es uno de las formas que define la cantidad de informaci\'on disponible para llevar a cabo la individualizaci\'on de filamentos. La importancia de este procedimiento radica en que a partir de la imagen es posible obtener una cantidad de caracter\'isticas de distinta \'indole, lo que permite en etapas posteriores clasificar de diversas formas nodos, aristas, de forma aislada o en conjuntos, efectivamente disminuyendo el espacio de b\'usqueda. Con las herramientas actuales disponibles en la literatura, es posible realizar la extracci\'on de una red de filamentos con algún nivel de informaci\'on como en \cite{xu2015soax}. Sin embargo, las transformaci\'on de aquella red a un grafo, as\'i como la incorporaci\'on de las caracter\'isticas y/o propiedades hacia el grafo son un procedimiento no automatizado, por lo que el esfuerzo que el experto debe realizar es cercano a individualizar los filamentos de manera manual.

A partir de las investigaciones en la literatura, es posible agrupar los m\'etodos para extraer la informaci\'on que permite la construcci\'on de un grafo a partir de una imagen, como lo son los nodos y las aristas, en dos conjuntos: Los que se basan en esqueletonizaci\'on\cite{lavado2018comparacion} y los que no. 

\subsection{Extracci\'on de un Grafo mediante Esqueletonizaci\'on}
\label{subsec:infoLossSkel}

%Que es la skeletonizacion y como extrae el grafo. Uso de liberia sknw
Los m\'etodos basados en esqueletonizaci\'on consisten primariamente en la reducci\'on de los p\'ixeles pertenecientes al plano de inter\'es o {\it foreground} en una imagen binaria, hasta formar una representaci\'on del objeto en la imagen de 1 p\'ixel de ancho. El proceso debe mantener la conectividad del objeto adelgazado y a su vez, reducir la dimensi\'on del objeto en la imagen para facilitar su an\'alisis\cite{saha2017skeletonization}. Un an\'alisis de los vecindarios de los p\'ixeles del esqueleto construido es una de las formas m\'as sencillas en que se puede distinguir si un p\'ixel representa un nodo o si es parte de un arista. Una librer\'ia que realiza tal an\'alisis es {\it skan}\cite{nunez2018new}, entregando estad\'isticas del grafo extra\'ido como largo promedio de una rama del esqueleto (equivalente a una arista del grafo), tipo de rama, curvatura de una rama, entre otras mediciones. Sin embargo, el formato de salida del grafo para esta herramienta corresponde a {\it Compressed sparse row} o CSR, lo que causa que un an\'alisis de mayor profundidad o el paso del grafo a una herramienta de individualizaci\'on de filamentos necesite de una librer\'ia adicional. 


Otra herramienta que realiza un an\'alisis similar para obtener un grafo a partir de un esqueleto es {\it sknw}, parte del framework {\it ImagePy}\cite{wang2018imagepy}. La diferencia propuesta por {\it sknw} radica en que se integra con la librer\'ia {\it NetworkX}\cite{hagberg2008exploring}, utilizando la estructura de datos para grafos que esta \'ultima posee para elegir entre m\'ultiples formatos de salida. Aquello otorga flexibilidad en la integraci\'on de herramientas que utilizan como base el grafo para realizar an\'alisis posteriores, como es el caso de la individualizaci\'on de filamentos.

% se puede extraer el skeleton con librerias en varios lenguajes, como octave, matlab o python (usando scipy). La opci\'on se encuentra en la secci\'on de herramientas morfol\'ogicasd de cada uno. De ahi al grafo, se puede utilizar un procedimiento que haga an\'alisis de vecindarios de p\'ixeles como \cite{qiu2014quantitative}

%problema de skeleton
Independiente de la herramienta usada para obtener la informaci\'on topol\'ogica de la c\'elula observada, el procedimiento de esqueletonizaci\'on puede sufrir de p\'erdida de informaci\'on, dado que mediante el adelgazamiento utilizado se rompe la relaci\'on entre los p\'ixeles que conforman el elemento que se adelgaza y los p\'ixeles que constituyen el esqueleto obtenido. La informaci\'on que relaciona los p\'ixeles vecinos a los p\'ixeles del esqueleto es relevante ya que puede reflejar informaci\'on geom\'etrica como ancho o puede ser \'util al obtener informaci\'on derivada mediante m\'etodos como {\it image moments}\cite{flusser2009moments}\cite{chaumette2004image}. En base a lo anterior, se desarrolla una herramienta que captura esta informaci\'on a partir de la imagen usada en el proceso de esqueletonizaci\'on. A su vez, con esta herramienta tambi\'en es posible construir un grafo de la red de filamentos, sin embargo, el resultado no es de alta calidad. El procedimiento consta de los siguientes pasos:
% explicar que el hecho de G completo mediante lo q la arista es al problema de identificación de filamentos
%presento el extractor aproximado de grafos, la noción de puntos cluster/superPixels/blobs y el centro de masa como representante, dado que se puede obtener de forma simple mediante los "image moments". Además, distintos niveles de image moments permiten obtener información adicional útil en la descripción del cluster.

\begin{enumerate}
    \item Se generan clusters, tambi\'en llamados {\it super pixels} mediante una agrupaci\'on en base a un kernel de tama\~no 3x3, que busca separar de forma local los pixels que corresponden a {\it background} y no aportan informaci\'on, con respecto a los que son parte de la c\'elula observada ({\it foreground}), siendo estos \'ultimos los que concentran el inter\'es para el an\'alisis. La creaci\'on de clusters permite establecer la primera informaci\'on de vecindarios.
    \item Cada cluster es representado por un nodo, ubicado en el centro de masa del {\it super pixel}. En conjunto con el centro de masa es posible obtener informaci\'on geom\'etrica mediante los {\it raw image moments}\cite{chaumette2004image} del cluster. En este punto tambi\'en se procede a realizar fusiones de clusters en base a 2 criterios: Aquellos clusters con un n\'umero de p\'ixeles inferior a un umbral definido por el par\'ametro {\it Max\_Thickness} y los clusters que s\'olo tengan 2 vecinos y se encuentren bajo un umbral de cercan\'ia definido por el par\'ametro {\it Connectivity\_Threshold}. El par\'ametro {\it Max\_Thickness} hace referencia al ancho promedio de un filamento en la imagen y es definido por el usuario, mientras que {\it Connectivity\_Threshold} es el resultado del m\'ultiplo entre {\it Max\_Thickness} y un factor que depende de la c\'elula observada, variando entre 0.4 y 0.5. Una de las ventajas de fusionar clusters bajo los criterios mencionados radica en identificar y eliminar ciclos triangulares entre vecinos, mejorando la informaci\'on de vecindarios, sirviendo como una heurística para limitar el n\'umero de nodos.
    
    %\item A partir de la informaci\'on de vecindario de los nodos definitivos obtenidos en el paso anterior, se generan aristas entre nodos vecinos. Se obtiene el largo e  informaci\'on angular para cada arista creada. %Esta \'ultima es utilizada m\'as adelante para definir puntos de partida de las hormigas.
\end{enumerate}

Para asociar la informaci\'on de los vecindarios creados mediante los clusters con la informaci\'on extra\'ida mediante la esqueletonizaci\'on, se realiza un procedimiento similar al del paso 2, donde los nodos que representan clusters son fusionados con los nodo del esqueleto. 
Utilizando la posici\'on del nodo del esqueleto como centro de una matrix de tama\~no 3x3, se obtienen los clusters a los que pertenecen estos p\'ixeles. los cuales son absorbidos por el nodo del esqueleto respectivo, formando un nuevo cluster. Este nuevo cluster reemplaza a los clusters absobidos, heredando los vecinos que estos ten\'ian, pero manteniendo la posici\'on del nodo del esqueleto.

%Se define como generador {\it aproximado} de grafos debido a que puede sufrir de peque\~nas desviaciones en la ubicaci\'on de los nodos y las aristas con respecto a lo que otras herramientas de esqueletonizaci\'on e identificaci\'on de intersecciones en conjunto pueden hacer en la misma tarea. Uno de los objetivos en la creaci\'on del generador {\it aproximado} de grafos se debe a que facilita el uso de im\'agenes encontradas en el estado del arte.

En el caso que no sea posible obtener grafo mediante la esqueletonizaci\'on de la imagen, esta herramienta considera un paso adicional al procedimiento de 2 pasos presentado previamente. El tercer paso consiste en generar aristas entre los nodos vecinos, en base a la informaci\'on de vecindario, obteniendo el largo e informaci\'on angular para cada arista creada. El grafo generado se denomina como un grafo aproximado, ya que puede presentar deformaciones.


\subsection{Obtenci\'on de Informaci\'on Adicional}

Para dar uso a la informaci\'on recuperada de acuerdo a lo expresado en la secci\'on \ref{subsec:infoLossSkel}, se analizaron diversos filtros \'utiles para describir estructuras alargadas, como {\it Gabor}, {\it Anistropic Diffusion} y Frangi para {\it Veselness}. El filtro Frangi para {\it Veselness}\cite{frangi1998multiscale}\cite{fu2018frangi}, cuantifica cuan alargada es una estructura ({\it veselness value}), en base a los eigenvectores y eigenvalores de la matriz Hessiana (ecuaci\'on \eqref{eq:HessianMat}) posterior a la aplicaci\'on de uno o varios filtros Gaussianos para suavizar una imagen. Este filtro es utilizado en la detecci\'on de estructuras alargadas como arterias y venas, pudiendo replicarse parcialmente mediante el an\'alisis de p\'ixeles con {\it image moments}\cite{flusser2009moments}. La posibilidad de replicar el filtro de Frangi, sin necesidad de configurar par\'ametros, es lo que llev\'o a elegirlo por sobre las otras opciones.

\begin{equation}
    \label{eq:HessianMat}
    H = \begin{bmatrix}
        H_{xx} & H_{xy} \\
        H_{xy} & H_{yy} 
        \end{bmatrix}
\end{equation}

Una respuesta de {\it veselness value} que denota una estructura alargada se obtiene si los 2 eigenvalores, $\lambda_1$ y $\lambda_2$ ($|\lambda_2| \geq |\lambda_1|$) satisfacen $|\lambda_1| \approx 0 $ y $|\lambda_2| \gg |\lambda_1|$. Los eigenvalores se obtienen mediante la ecuaci\'on \ref{eq:lambdaFrangi}.

\begin{equation}
    \label{eq:lambdaFrangi}
    \lambda_{1,2} = \dfrac{(H_{xx} + H_{yy}) \pm \sqrt{(H_{xx} - H_{yy})^{2} + 4\cdot H_{xy}^{2}     } }{2}
\end{equation}

Otra forma de obtener los valores de lambda de la ecuaci\'on \ref{eq:lambdaFrangi} es utilizando los {\it central image moments} o momentos centrales, que derivan de los {\it raw image moments} obtenidos en el segundo paso la herramienta para recuperar informaci\'on presentada en la secci\'on \ref{subsec:infoLossSkel}. Se define un {\it raw image moment} de orden $p+q$ para una imagen en la ecuaci\'on \eqref{eq:rawImageMoment}, donde $f(x,y)$ corresponde a la intensidad de la imagen en un punto (x,y). El {\it raw moment} $M_{00}$ refleja la "masa" de la imagen, correspondiendo al \'area o volumen si se trata de una imagen binaria. 

Para el c\'alculo de los momentos centrales se agregan los componentes del centroide, $\overline{x}$ e $\overline{y}$, basados en los {\it raw moments}, como indican las ecuaciones \eqref{eq:avgFromRawMomts} y \eqref{eq:centralImageMoment}.

\begin{subequations}
\begin{equation}
    \label{eq:rawImageMoment}
    M_{pq} = \sum\limits_{x} \sum\limits_{y} x^p \cdot y^q \cdot f(x,y)
\end{equation}
\begin{equation}
    \label{eq:avgFromRawMomts}
    \overline{x} = \frac{M_{10}}{M_{00}}, \quad
    \overline{y} = \frac{M_{01}}{M_{00}}
\end{equation}
\begin{equation}
    \label{eq:centralImageMoment}
    \mu_{pq} = \sum\limits_{x} \sum\limits_{y} (x - \overline{x})^{p} \cdot (y - \overline{y})^{q} \cdot f(x,y)
\end{equation}
\end{subequations}

As\'i, es posible construir una matriz de covarianza, equivalente a la matriz hessiana en la ecuaci\'on \eqref{eq:HessianMat}, utilizando los momentos centrales de segundo orden, $\mu_{20}$, $\mu_{02}$ y $\mu_{11}$ divididos por el momento central de orden cero $\mu_{00}$ (ecuaciones \eqref{eq:mu20}, \eqref{eq:mu02} y \eqref{eq:mu11}), obteniendo los eigenvalores mediante la ecuaci\'on \eqref{eq:lambdaMoments}.

\begin{subequations}
\begin{align}
    \mu_{20}^{\prime} &= \frac{\mu_{20}}{\mu_{00}} = \frac{M_{20}}{M_{00}} - \overline{x}^{2} \label{eq:mu20} \\
    \mu_{02}^{\prime} &= \frac{\mu_{02}}{\mu_{00}} = \frac{M_{02}}{M_{00}} - \overline{y}^{2} \label{eq:mu02} \\
    \mu_{11}^{\prime} &= \frac{\mu_{11}}{\mu_{00}} = \frac{M_{11}}{M_{00}} - \overline{x}\cdot\overline{y} \label{eq:mu11}
\end{align}

\begin{equation}
    \label{eq:covMatLambda}
    cov[f(x,y)] = \begin{bmatrix}
        \mu_{20}^{\prime} & \mu_{11}^{\prime} \\
        \mu_{11}^{\prime} & \mu_{02}^{\prime} 
        \end{bmatrix}
\end{equation}

\begin{equation}
    \label{eq:lambdaMoments}
    \lambda_{1,2} = \dfrac{(\mu_{20}^{\prime} + \mu_{02}^{\prime}) \pm \sqrt{(\mu_{20}^{\prime} - \mu_{02}^{\prime})^{2} + 4\cdot \mu\prime_{11}^{2} }}{2}
\end{equation}
\end{subequations}

Con los valores de $\lambda$, es posible calcular caracter\'isticas de una estructura alargada como su excentricidad o su eje principal de inercia. Estas medidas pueden ayudar a mejorar la clasificaci\'on de segmentos del grafo durante la identificaci\'on de filamentos.


% informacion geometrica mediante calculo de angulos entre aristas
Un manera adicional de generar informaci\'on que facilite la discriminaci\'on de secciones del grafo es a trav\'es del c\'alculo de los \'angulos entre las aristas del grafo. Esto se relaciona a criterio de rectitud que tienen los filamentos, que var\'ia dependiendo de la c\'elula a la que pertenezca. Este comportamiento de los filamentos permite delimitar el \'angulo m\'aximo que 2 aristas contiguas pueden tener para ser considerados parte del mismo filamento, denomin\'andose este umbral como $Max\_Angle$. Cualquier valor por sobre $Max\_Angle$ permite descartar de forma absoluta esa combinanci\'on de aristas para un mismo filamento. 


A su vez, este criterio posee un segundo umbral, definido como $\theta$  que define el \'angulo m\'aximo bajo el que se considera que 2 aristas contiguas respetan con certeza la rectitud necesaria para formar parte del mismo filamento. Es decir, si 2 aristas contiguas forman un \'angulo en el rango $[0, \theta]$, deben ser parte del mismo filamento. El rango entre ambos umbrales, $]\theta, Max\_Angle]$ delimita los pares de aristas que a priori no representan combinaciones que respetan el criterio de rectitud, pero cuya explicaci\'on puede encontrarse en variaciones inducidas durante la extracci\'on del grafo desde la imagen, por lo que es necesario incorporar la exploraci\'on de estos pares de aristas.


Finalmente, para el caso de los nodos, el an\'alisis del grado de cada uno permite identificar la existencia de ciclos\cite{wilson1979introduction} en un filamento. La propiedad de un filamento de poder o no tener un ciclo es informaci\'on disponible priori que depende del tipo de c\'elula observada, permitiendo limitar posibles asociaciones entre nodos. En el caso particular de no permitir ciclos, un filamento no podr\'ia pasar m\'as de una vez por cada nodo que lo conforma. 

%se destaca dentro de los criterios: busca evitar perder información de la imagen/ampliar la informacion así como tener un costo computacional bajo, en conjunto con disminuir la interacción del usuario. 


% ambas opciones generan degeneraciones/deformaciones q afectan
\section{Exploraci\'on del Espacio de Soluciones}

La b\'usqueda de conjuntos de nodos o aristas adyacentes (denominados caminos) en un grafo, para individualizar uno o m\'as filamentos, constituye un espacio de soluciones que no es posible de recorrer en tiempo polinomial, dado que sin restricciones las combinaciones crecen exponencialmente\cite{buchin2007number}\cite{biswas2012hamiltonian}. Un planteamiento similar a lo anterior es el {\it Path Cover Problem} o PCP, en el que se descompone un grafo dirigido en caminos con el objetivo de obtener un conjunto de caminos. Cada nodo o arista debe pertenecer exactamente a un camino y los caminos pueden comenzar o terminar en cualquier parte del grafo.

En el caso de la individualizaci\'on de filamentos, es necesario que la definici\'on del problema considere como v\'alidos caminos que representen los casos de filamentos con o sin ciclos, as\'i como superposici\'on y/o cruce. Lo anterior impide forzar la pertenencia de un nodo o arista a un s\'olo camino. 

% explicar define: como lo hace define? y como se observa la potencial perdida de soluciones factibles al usar un solo criterio
La investigaci\'on de \cite{breuer2015define}, denominada {\tt DeFiNe}, descrita brevemente en el cap\'itulo \ref{chap:stateoftheart} presenta el {\it Filament Cover Problem} o FCP como una extensi\'on del PCP, flexibilizando la pertenencia de cada arista a al menos un camino. Adem\'as plantean como funci\'on objetivo la minimizaci\'on de la diferencia de la homogeneidad (aspereza) entre las aristas que componen un camino. Para acotar el espacio de soluciones, los autores de DeFiNe proponen 2 heur\'isticas fundamentando que el FCP en \'arboles en vez de grafos es soluble en tiempo polinomial. Aquello se basa en lo definido por \cite{lin2006vertex} que indica que para un problema de {\it covering}, existe un {\it Set System} $(S,C)$, donde S es el conjunto total y finito de sets, y C es un conjunto de subsets pertenecientes a S. En el caso espec\'ifico del {\it Minimum Set Cover} (MSC), el objetivo es encontrar un subconjunto $C'$ de $C$ tal que cada elemento de $S$ pertenezca al menos 1 vez a uno de los miembros de $C'$.

%Un grafo completamente conectado puede tener n(n-1)/2 aristas, lo que para un n muy grande puede implicar un costo computacional alto
%A su vez, otro motivo para evitar que $G$ sea un grafo completo radica en que para los filamentos observados en la naturaleza no es una condici\'on comun... no encuentro la fuente de esto

%Denominamos todas estas opciones de caminos como el conjunto $P$, del cual debemos extraer un subconjunto $P'$ mediante una estrategia que permita realizar esto en tiempo polinomial independientemente de la cantidad de aristas del grafo.

Para un {\it set system} $(S,C)$ que pueda ser representando por un \'arbol $T$, es posible mutar la definici\'on de $S$ al conjunto de nodos que componen un grafo $G$, y a su vez, hacer que cada subset $c \in C$ represente un camino simple en $G$. Un camino simple es equivalente a un \'arbol simple (2 nodos son unidos por a lo m\'as 1 arista) y ac\'iclico.


Se destaca en \cite{lin2006vertex} que no todos los caminos simples de $G$ est\'an representados en $C$. Esta representaci\'on de {\it covering} de caminos, o {\it Path Cover}, es denominada {\it Vertex Covering by Paths on Graphs} (VcpG) y difiere de un {\it Path Cover} tradicional al permitir caminos que compartan nodos. Luego, para el caso de \'arboles, VcpG se renombra a VcpT ({\it Vertex Covering by Paths on Trees}), permitiendo el reemplazo de nodos por aristas sin generar cambios significativos en el planteamiento del problema. Esta modificaci\'on de nodos a aristas se denomina EcpT ({\it Edge Covering by Paths on Trees}). %recordar mas adelante como posible perdida de soluciones factibles


Lo anterior establece el fundamento para que los autores de DeFiNe\cite{breuer2015define} utilicen EcpT como base, teniendo un m\'aximo de caminos $n(n)-1/2 = \mathcal{O}(n^{2})$, con una complejidad $\mathcal{O}(n^{4})$ para obtener esos caminos. La obtenci\'on de caminos a partir de un \'arbol $T$ que contiene los nodos del grafo $G$ se realiza mediante la divisi\'on continua del \'arbol en m\'ultiples bosques, hasta que los bosques resultantes sean s\'olo caminos simples. Sin embargo, los caminos resultantes no comparten nodos o aristas, es decir, no se superponen. Para solucionar aquello manteniendo la resoluci\'on del FCP en tiempo polinomial, se agrega en DeFiNe el par\'ametro $k$, que define el n\'umero m\'aximo de superposiciones de caminos en una arista. La complejidad con el nuevo par\'ametro queda como $\mathcal{O}(n^{2k+2})$.


%aca entran las formas de obtener esos arboles simples y bosques, con las heuristicas de define
Dado que el conjunto total de caminos en un grafo, definido como $P$, no es calculables en tiempo polinomial, DeFine propone 2 heur\'isticas para construir \'arboles de los cuales se puedan extraer un conjunto representativo de caminos simples, definidos como $P'$. Cada una de las heur\'isticas, {\it BFS} y {\it RMST}, explicadas en el cap\'itulo \ref{chap:stateoftheart}, proveen un conjunto $P'$, al que se le aplica la funci\'on objetivo, para encontrar los miembros de $p \in P'$ que mejor minimicen la diferencia de homogeneidad en sus caminos. La restricci\'on de lo anterior es que cada arista del grafo pertenezca a lo menos a un camino. 

%\begin{equation}
%p = (e_1, e_2,..., e_n)\\
%p = ((v_1,v_2), (v_2,v_4),..., (v_n-1,v_n))
%\label{eq:path}
%\end{equation}

% Comparándose con define, Problema 1 caminos validos no llegan a P'

La generaci\'on del conjunto $P'$ planteado en DeFiNe puede llevar a excluir caminos v\'alidos al realizar la representaci\'on de un {\it set system} mediante un \'arbol, o al utilizar una sola propiedad asociada a un filamento en una de sus heur\'isticas. 


El presente trabajo plantea en el cap\'itulo \ref{sec:modeloOpti} un modelo de optimizaci\'on que permite generar caminos utilizando m\'as de una propiedad asociada a un filamento o al grafo que representa la red de filamentos. Esto permite individualizar filamentos en casos de superposici\'on, cruce o discriminando si el filamento puede tener o no ciclos. Adem\'as se presentan diversas heur\'isticas para reducir el tama\~no del espacio de soluciones.

%Ejemplo camino verde en Spinning Marchantia (imagenes)



%\chapter{Conclusiones}
%\begin{conclusion}
\label{chap:conclu}
La individualizaci\'on autom\'atica de filamentos es un problema que se encuentra limitado por la extracci\'on de informaci\'on que se pueda realizar desde una imagen. Entre las limitantes de este paso previo se puede considerar la resoluci\'on y el ruido de la imagen, as\'i como la capacidad de la herramienta que genera un grafo a partir de la misma imagen, de conservar la estructura topol\'ogica con la menor cantidad de discontinuidades. Bajo estas consideraciones se debe  obtener la mayor cantidad de informaci\'on en pos de lograr un mejor resultado.

%Diversos procedimientos de limpieza realizados sobre la imagen de una c\'elula o muestra pueden 

%Los procedimientos de limpieza sobre la imagen de la c\'elula observada, as\'i como el proceso de esqueletonizaci\'on pueden causar perdida de informaci\'on o introducir deformaciones, impactando en la calidad del grafo que se extrae, y que sirve de base para la individualizaci\'on de filamentos, por lo que resulta imperativo obtener la mayor cantidad de informaci\'on en estos pasos para compensar estos posibles problemas. 

La obtenci\'on de m\'ultiples caracter\'isticas de los filamentos es cr\'itica para descartar rapidamente soluciones de baja calidad, algo especialmente necesario en un problema en el que la combinaci\'on de aristas que conforman un camino que representa un filamento puede crecer exponencialmente. Estas caracter\'isticas pueden provenir no s\'olo de aspectos geom\'etricos de los filamentos, sino que tambi\'en de informaci\'on topol\'ogica y/o espacial de los mismos.
A\'un con aquella informaci\'on que permite explorar de mejor forma las combinaciones de aristas contiguas o caminos que representan un filamento en un grafo, la informaci\'on de la c\'elula observada disponible {\it a priori}, como el grado m\'aximo de los nodos o la presencia de ciclos en los filamentos, facilita el establecimiento de umbrales y/o cotas que describen parcialmente aspectos del comportamiento de los filamentos.

El an\'alisis derivado de las m\'ultipes caracter\'isticas se enfoca en obtener informaci\'on cuantitativa respecto al comportamiento de los filamentos, marcando una diferencia con otras investigaciones con an\'alisis de tipo cualitativo. De forma similar, el algoritmo propuesto cuenta con una estructura flexible que permite incorporar informaci\'on y/o l\'ogicas adicionales, las que dependiendo de la etapa en la metaheur\'istica ACO en que se encuentren, pueden variar la forma en que se explora el espacio de b\'usqueda o en que se aplican restricciones a los resultados parciales obtenidos. Otro aspecto relativo a la flexibilidad del algoritmo propuesto radica en contar con  par\'ametros predefinidos dependiendo del tipo de c\'elula, permitiendo la personalizaci\'on de estos par\'ametros por parte del experto, lo que busca facilitar el an\'alisis y a su vez permite automatizar la extracci\'on de informaci\'on para los mismos tipos de c\'elulas.


% se prueba la hipótesis, se obtienen mejores resultados que el estado del arte pero aun falta
Los resultados obtenidos en esta investigaci\'on incorporan los elementos mencionados previamente, permitiendo individualizar filamentos de forma similar o mejor con respecto a m\'etodos en el estado del arte, y a su vez cumpliendo con la hip\'otesis de identificar cada filamento resolviendo un modelo de optimizaci\'on. A pesar de no lograr una individualizaci\'on correcta de la totalidad de los filamentos, se obtiene un comportamiento estable en las distintas iteraciones de cada evaluaci\'on, as\'i como se logra un n\'umero de filamentos propuestos que en la mayor\'ia de las ocasiones se acerca al n\'umero de filamentos indicado por el experto. Cabe volver a destacar lo mencionado en el cap\'itulo \ref{sec:genGrafFromImage} en relaci\'on a que es posible que 2 expertos indiquen diferentes filamentos a partir de la misma imagen, por lo que es necesario incorporar un mayor n\'umero de pruebas a este trabajo para descartar la existencia de un sobre-ajuste del modelo a la informaci\'on provista por los expertos que realizaron las individualizaciones manuales de las im\'agenes utilizadas.


Con respecto al uso de la metaheur\'istica ACO como base del algoritmo propuesto, esta propone una diferencia con respecto del planteamiento de PCP o FCP, el que obliga a cada arista a pertenecer al menos a un camino. En cambio, el algoritmo propuesto relaja aquella restricci\'on dado que pueden existir aristas aisladas producto una extracci\'on de grafo con perdida de informaci\'on, ocasionando discontinuidades. La captaci\'on de aquella informaci\'on es relevante dado que existen m\'etodos en el estado del arte que permiten realizar la uni\'on de una arista aislada y el filamento al que potencialmente puede pertenecer, pudiendo incorporarse como etapas adicionales al modelo. 

%cumplimiento de objetivos
En general, el algoritmo propuesto permite resolver un modelo de optimizaci\'on para individualizar filamentos a partir de un grafo con pesos que representa una red de filamentos, cumpliendo con lo indicado como primer objetivo espec\'ifico. En cuanto al cumplimiento de los dem\'as objetivos espec\'ificos, se tiene:
\begin{itemize}
    \item {\bf Implementar un algoritmo que resuelva el modelo de optimizaci\'on, entregando como salida la identificaci\'on de filamentos, considerando casos de solapamiento y/o cruce}: El algoritmo propuesto cumple con este objetivo, de acuerdo a lo presentado en los cap\'itulos \ref{sec:modeloOpti} y \ref{chap:res}.
    
    \item {\bf Identificar la ponderaci\'on de propiedades que entregue mejores resultados para grafos que representen una neurona, una bacteria y una c\'elula eucariota de planta:}: Se realiza una ponderaci\'on de las propiedades utilizadas en el algoritmo propuesto en la secci\'on \ref{subsec:ponderacion}, la que var\'ia dependiendo del tipo de c\'elula y se encuentra asociado a la distribuci\'on de las caracter\'isticas en las distintas etapas de la metaheur\'istica ACO.
    
    \item {\bf Evaluar t\'ecnicas en el estado del arte que realicen individualizaci\'on de filamentos.:} Entre Las t\'ecnicas descritas en el cap\'itulo \ref{chap:stateoftheart}, solo el m\'etodo desarrollado por \citepxl{breuer2015define} realiza individualizaci\'on de filamentos, siendo el m\'etodo evaluado, en conjunto con el algoritmo propuesto, en el cap\'itulo \ref{chap:res}.
    
    %Aun cuando las herramientas que describen las t\'ecnicas evaluadas no realizan individualizaci\'on de filamentos y se enfocan en la extracci\'on de informaci\'on geom\'etrica y/o topol\'ogica. 
    %Sin embargo no realizan individualizaci\'on de filamentos, por lo que la evaluaci\'on fue dirigida hacia la investigaci\'on que da lugar a DeFiNe, que si realiza este procedimiento, siendo evaluada en el cap\'itulo \ref{chap:res}.
\end{itemize}

Mediante el cumplimiento del objetivo general as\'i como de los objetivos espec\'ificos, se tiene que el algoritmo propuesto permite aceptar la hip\'otesis planteada al comienzo de esta investigaci\'on.

Como cotinuaci\'on de esta investigaci\'on se encuentra la opci\'on de intensificar la extracci\'on de informaci\'on topol\'ogica y espacial que permita realizar la asociaci\'on de secciones importantes de los filamentos o elementos alrededor de estos en la c\'elula observada, con el fin de seguir a\~nadiendo informaci\'on al modelo. Tambi\'en es posible vislumbrar una extensi\'on de este trabajo en la identificaci\'on de las interacciones entre los filamentos individualizados. Lo anterior se refleja en microt\'ubulos como la identificaci\'on de interacciones tipo cat\'astrofe, zippering o nucleaci\'on, mientras que para una neurona ser\'ia la clasificaci\'on autom\'atica de los filamentos en axon o dendritas primarias o secundarias.

% \input{glosario.tex} % opcional

\bibliographystyle{plain}
\bibliography{bibliografia}

% \input{anexo_apendices.tex} % opcionales

\end{document}
