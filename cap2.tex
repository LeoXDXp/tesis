%\chapter{Problem\'atica de la Individualizaci\'on de Filamentos a partir de un Grafo}
\chapter{Antecedentes}
\label{chap:cap2}

%problema previo, o problema 0 que consiste en la generaci\'on de un grafo a partir de una imagen que contenga una red, que en este caso, representaria a una red de filamentos. 

En base lo expuesto en el cap\'itulo \ref{chap:stateoftheart}, para el enfoque de utilizar grafos para la individualizaci\'on de filamentos existe una brecha entre la obtenci\'on del grafo y su an\'alisis. Este paso previo, la extracci\'on de un grafo a partir de una imagen, consiste en extraer un grafo $G = (V,E)$ de una imagen tal que $G$ sea un grafo simple, no dirigido, ponderado, conectado o desconectado, con o sin ciclos. Esto implica que exista a lo m\'as 1 arista por cada par de nodos adyacentes, prohibi\'endose la existencia de nodos conectados consigo mismos. Se definen los v\'ertices/nodos del grafo $G$ como $V(G)$ y las aristas de $G$ como $E(G)$. 
Que el grafo $G$ sea ponderado implica que para las aristas del grafo ($E(G)$), existen caracter\'isticas asociadas que se expresan como caracter\'isticas geom\'etricas, topol\'ogicas, espaciales y/u otras. Es importante evitar que $G$ sea un grafo completo, dado que con n nodos/v\'ertices $G$ llega a tener $\frac{n(n-1)}{2}$ aristas.
%$\forall e \in \quad E(G) \quad  \exists $ 

Algunas de las dificultades involucradas en la extracci\'on de informaci\'on a partir de una imagen se encuentran en los m\'etodos presentados en el cap\'itulo \ref{chap:stateoftheart}, dentro de las que destacan el ruido y la resoluci\'on. Un ejemplo de aquello se observa en la figura \ref{fig:NoConsenso}.

\begin{figure*}[h]
    \begin{tabular}{c c c}
        \multirow[c]{2}{*}[2.5cm]{
        \begin{subfigure}[t]{0.4\textwidth}
        \includegraphics[scale=0.5]{imagenes/NoConsenso.png}
        \caption{Microt\'ubulos en planta {\it Marchantia}.\\Fuente: Paula Llanos}
        \label{fig:NoConsensoGeneral}
        \end{subfigure}  
        }
        &
        \multirow[c]{2}{*}[2cm]{
        \begin{subfigure}[t]{0.25\textwidth}
        \includegraphics[]{imagenes/NoConsenso2.png}
        \caption{Secci\'on resaltada en rojo de \ref{fig:NoConsensoGeneral}}
        \label{fig:NoConsensoRect}
        \end{subfigure}
        }
        &
        \begin{subfigure}[t]{0.21\textwidth}
        \includegraphics[scale=0.8]{imagenes/NoConsenso3.png}
        \caption{Opci\'on 1 de microt\'ubulos en \ref{fig:NoConsensoRect}}
        \label{fig:NoConsensoOpcion1}
        \end{subfigure} \\
        & &
        \begin{subfigure}[b]{0.21\textwidth}
        \includegraphics[scale=0.8]{imagenes/NoConsenso4.png}
        \caption{Opci\'on 2 de microt\'ubulos en \ref{fig:NoConsensoRect}}
        \label{fig:NoConsensoOpcion2}
        \end{subfigure} \\
    \end{tabular}
    
    \caption{Dificultad de individualizaci\'on que enfretan los expertos al analizar manualmente una imagen de filamentos, en particular, microt\'ubulos.}
    \label{fig:NoConsenso}
\end{figure*}

Mientras que el ruido ha sido estudiado en la literatura, el problema de resoluci\'on de filamentos depende principalmente de la capacidad del microscopio que se utilice. Como se menciona en la introducci\'on, el l\'imite m\'aximo de resoluci\'on, denominado $\frac{\lambda}{2}$, determina el tama\~no m\'inimo que 2 objetos que se encuentren juntos pueden tener para no observarse como un \'unico elemento. Lo anterior sucede para algunos tipos de filamentos como los microt\'ubulos que pueden medir tan solo 25 nan\'ometros, lo que se encuentra por debajo de $\frac{\lambda}{2}$ para diversos microscopios. 



Una vez obtenido el grafo que representa una red de filamentos, el problema siguiente se encuentra en el que 2 expertos pueden discernir con respecto a los filamentos identificables en una imagen, por lo que no es posible conocer a priori del origen y el final de un filamento, para las resoluciones actuales de las imagenes obtenidas a partir de la microscop\'ia. Una dificultad adicional se presenta en la representaci\'on de un filamento en un grafo, ya que esta se basa en un conjunto de aristas adyacentes, denominadas caminos o recorridos, lo que lleva a tener un universo de hasta $n!$ posibles combinaciones en el espacio de soluciones en el que se busca un camino.

A partir del problema anterior, el problema final en la individualizaci\'on de filamentos lo constituye la elecci\'on del subconjunto de caminos, que debe ser seleccionado entre el total de caminos que representan soluciones factibles. Esto implica que el problema no solo sea un problema combinatorial de generar soluciones factibles a partir del conjunto de aristas, sino que adem\'as debe considerar la discriminaci\'on entre estos para obtener el subconjunto de mayor calidad, pudiendo representarse como un problema de optimizaci\'on combinatorial.

Los 3 problemas presentados se formalizan a continuaci\'on.

%problema previo
%presentar el problema previo, como un puente necesario en la automatizaci\'on de la extracci\'on,  para analizar un grafo que representa la red de filamentos, que de lo contrario tendria que ser realizado a mano, implicando que la persona realizando el análisis podría llevar a cabo la individualizaci\'on de filamentos en el mismo acto.

\section{Generaci\'on de un Grafo desde una Imagen}
La extraci\'on o generaci\'on de un grafo que representa una red de filamentos a partir de una imagen es uno de las formas que define la cantidad de informaci\'on disponible para llevar a cabo la individualizaci\'on de filamentos. La importancia de este procedimiento radica en que a partir de la imagen es posible obtener una cantidad de caracter\'isticas de distinta \'indole, lo que permite en etapas posteriores clasificar de diversas formas nodos, aristas, de forma aislada o en conjuntos, efectivamente disminuyendo el espacio de b\'usqueda. Con las herramientas actuales disponibles en la literatura, es posible realizar la extracci\'on de una red de filamentos con algún nivel de informaci\'on como en \cite{xu2015soax}. Sin embargo, las transformaci\'on de aquella red a un grafo, as\'i como la incorporaci\'on de las caracter\'isticas y/o propiedades hacia el grafo son un procedimiento no automatizado, por lo que el esfuerzo que el experto debe realizar es cercano a individualizar los filamentos de manera manual.

A partir de las investigaciones en la literatura, es posible agrupar los m\'etodos para extraer la informaci\'on que permite la construcci\'on de un grafo a partir de una imagen, como lo son los nodos y las aristas, en dos conjuntos: Los que se basan en esqueletonizaci\'on\cite{lavado2018comparacion} y los que no. 

\subsection{Extracci\'on de un Grafo mediante Esqueletonizaci\'on}
\label{subsec:infoLossSkel}

%Que es la skeletonizacion y como extrae el grafo. Uso de liberia sknw
Los m\'etodos basados en esqueletonizaci\'on consisten primariamente en la reducci\'on de los p\'ixeles pertenecientes al plano de inter\'es o {\it foreground} en una imagen binaria, hasta formar una representaci\'on del objeto en la imagen de 1 p\'ixel de ancho. El proceso debe mantener la conectividad del objeto adelgazado y a su vez, reducir la dimensi\'on del objeto en la imagen para facilitar su an\'alisis\cite{saha2017skeletonization}. Un an\'alisis de los vecindarios de los p\'ixeles del esqueleto construido es una de las formas m\'as sencillas en que se puede distinguir si un p\'ixel representa un nodo o si es parte de un arista. Una librer\'ia que realiza tal an\'alisis es {\it skan}\cite{nunez2018new}, entregando estad\'isticas del grafo extra\'ido como largo promedio de una rama del esqueleto (equivalente a una arista del grafo), tipo de rama, curvatura de una rama, entre otras mediciones. Sin embargo, el formato de salida del grafo para esta herramienta corresponde a {\it Compressed sparse row} o CSR, lo que causa que un an\'alisis de mayor profundidad o el paso del grafo a una herramienta de individualizaci\'on de filamentos necesite de una librer\'ia adicional. 


Otra herramienta que realiza un an\'alisis similar para obtener un grafo a partir de un esqueleto es {\it sknw}, parte del framework {\it ImagePy}\cite{wang2018imagepy}. La diferencia propuesta por {\it sknw} radica en que se integra con la librer\'ia {\it NetworkX}\cite{hagberg2008exploring}, utilizando la estructura de datos para grafos que esta \'ultima posee para elegir entre m\'ultiples formatos de salida. Aquello otorga flexibilidad en la integraci\'on de herramientas que utilizan como base el grafo para realizar an\'alisis posteriores, como es el caso de la individualizaci\'on de filamentos.

%problema de skeleton
Independiente de la herramienta usada para obtener la informaci\'on topol\'ogica de la c\'elula observada, el procedimiento de esqueletonizaci\'on puede sufrir de p\'erdida de informaci\'on, dado que mediante el adelgazamiento utilizado se rompe la relaci\'on entre los p\'ixeles que conforman el elemento que se adelgaza y los p\'ixeles que constituyen el esqueleto obtenido. La informaci\'on que relaciona los p\'ixeles vecinos a los p\'ixeles del esqueleto es relevante ya que puede reflejar informaci\'on geom\'etrica como ancho o puede ser \'util al obtener informaci\'on derivada mediante m\'etodos como {\it image moments}\cite{flusser2009moments}\cite{chaumette2004image}. En base a lo anterior, se desarrolla una herramienta que captura esta informaci\'on a partir de la imagen usada en el proceso de esqueletonizaci\'on. A su vez, con esta herramienta tambi\'en es posible construir un grafo de la red de filamentos, sin embargo, el resultado no es de alta calidad. El procedimiento consta de los siguientes pasos:
% explicar que el hecho de G completo mediante lo q la arista es al problema de identificación de filamentos
%presento el extractor aproximado de grafos, la noción de puntos cluster/superPixels/blobs y el centro de masa como representante, dado que se puede obtener de forma simple mediante los "image moments". Además, distintos niveles de image moments permiten obtener información adicional útil en la descripción del cluster.

\begin{enumerate}
    \item Se generan clusters, tambi\'en llamados {\it super pixels} mediante una agrupaci\'on en base a un kernel de tama\~no 3x3, que busca separar de forma local los pixels que corresponden a {\it background} y no aportan informaci\'on, con respecto a los que son parte de la c\'elula observada ({\it foreground}), siendo estos \'ultimos los que concentran el inter\'es para el an\'alisis. La creaci\'on de clusters permite establecer la primera informaci\'on de vecindarios.
    \item Cada cluster es representado por un nodo, ubicado en el centro de masa del {\it super pixel}. En conjunto con el centro de masa es posible obtener informaci\'on geom\'etrica mediante los {\it raw image moments}\cite{chaumette2004image} del cluster. En este punto tambi\'en se procede a realizar fusiones de clusters en base a 2 criterios: Aquellos clusters con un n\'umero de p\'ixeles inferior a un umbral definido por el par\'ametro {\it Max\_Thickness} y los clusters que s\'olo tengan 2 vecinos y se encuentren bajo un umbral de cercan\'ia definido por el par\'ametro {\it Connectivity\_Threshold}. El par\'ametro {\it Max\_Thickness} hace referencia al ancho promedio de un filamento en la imagen y es definido por el usuario, mientras que {\it Connectivity\_Threshold} es el resultado del m\'ultiplo entre {\it Max\_Thickness} y un factor que depende de la c\'elula observada, variando entre 0.4 y 0.5. Una de las ventajas de fusionar clusters bajo los criterios mencionados radica en identificar y eliminar ciclos triangulares entre vecinos, mejorando la informaci\'on de vecindarios, sirviendo como una heurística para limitar el n\'umero de nodos.
    
    %\item A partir de la informaci\'on de vecindario de los nodos definitivos obtenidos en el paso anterior, se generan aristas entre nodos vecinos. Se obtiene el largo e  informaci\'on angular para cada arista creada. %Esta \'ultima es utilizada m\'as adelante para definir puntos de partida de las hormigas.
\end{enumerate}

Para asociar la informaci\'on de los vecindarios creados mediante los clusters con la informaci\'on extra\'ida mediante la esqueletonizaci\'on, se realiza un procedimiento similar al del paso 2, donde los nodos que representan clusters son fusionados con los nodo del esqueleto. 
Utilizando la posici\'on del nodo del esqueleto como centro de una matrix de tama\~no 3x3, se obtienen los clusters a los que pertenecen estos p\'ixeles. los cuales son absorbidos por el nodo del esqueleto respectivo, formando un nuevo cluster. Este nuevo cluster reemplaza a los clusters absobidos, heredando los vecinos que estos ten\'ian, pero manteniendo la posici\'on del nodo del esqueleto.

%Se define como generador {\it aproximado} de grafos debido a que puede sufrir de peque\~nas desviaciones en la ubicaci\'on de los nodos y las aristas con respecto a lo que otras herramientas de esqueletonizaci\'on e identificaci\'on de intersecciones en conjunto pueden hacer en la misma tarea. Uno de los objetivos en la creaci\'on del generador {\it aproximado} de grafos se debe a que facilita el uso de im\'agenes encontradas en el estado del arte.

En el caso que no sea posible obtener grafo mediante la esqueletonizaci\'on de la imagen, esta herramienta considera un paso adicional al procedimiento de 2 pasos presentado previamente. El tercer paso consiste en generar aristas entre los nodos vecinos, en base a la informaci\'on de vecindario, obteniendo el largo e informaci\'on angular para cada arista creada. El grafo generado se denomina como un grafo aproximado, ya que puede presentar deformaciones.


\subsection{Obtenci\'on de Informaci\'on Adicional}

Para dar uso a la informaci\'on recuperada de acuerdo a lo expresado en la secci\'on \ref{subsec:infoLossSkel}, se analizaron diversos filtros \'utiles para describir estructuras alargadas, como {\it Gabor}, {\it Anistropic Diffusion} y Frangi para {\it Veselness}. El filtro Frangi para {\it Veselness}\cite{frangi1998multiscale}\cite{fu2018frangi}, cuantifica cuan alargada es una estructura ({\it veselness value}), en base a los eigenvectores y eigenvalores de la matriz Hessiana (ecuaci\'on \eqref{eq:HessianMat}) posterior a la aplicaci\'on de uno o varios filtros Gaussianos para suavizar una imagen. Este filtro es utilizado en la detecci\'on de estructuras alargadas como arterias y venas, pudiendo replicarse parcialmente mediante el an\'alisis de p\'ixeles con {\it image moments}\cite{flusser2009moments}. La posibilidad de replicar el filtro de Frangi, sin necesidad de configurar par\'ametros, es lo que llev\'o a elegirlo por sobre las otras opciones.

\begin{equation}
    \label{eq:HessianMat}
    H = \begin{bmatrix}
        H_{xx} & H_{xy} \\
        H_{xy} & H_{yy} 
        \end{bmatrix}
\end{equation}

Una respuesta de {\it veselness value} que denota una estructura alargada se obtiene si los 2 eigenvalores, $\lambda_1$ y $\lambda_2$ ($|\lambda_2| \geq |\lambda_1|$) satisfacen $|\lambda_1| \approx 0 $ y $|\lambda_2| \gg |\lambda_1|$. Los eigenvalores se obtienen mediante la ecuaci\'on \ref{eq:lambdaFrangi}.

\begin{equation}
    \label{eq:lambdaFrangi}
    \lambda_{1,2} = \dfrac{(H_{xx} + H_{yy}) \pm \sqrt{(H_{xx} - H_{yy})^{2} + 4\cdot H_{xy}^{2}     } }{2}
\end{equation}

Otra forma de obtener los valores de lambda de la ecuaci\'on \ref{eq:lambdaFrangi} es utilizando los {\it central image moments} o momentos centrales, que derivan de los {\it raw image moments} obtenidos en el segundo paso la herramienta para recuperar informaci\'on presentada en la secci\'on \ref{subsec:infoLossSkel}. Se define un {\it raw image moment} de orden $p+q$ para una imagen en la ecuaci\'on \eqref{eq:rawImageMoment}, donde $f(x,y)$ corresponde a la intensidad de la imagen en un punto (x,y). El {\it raw moment} $M_{00}$ refleja la "masa" de la imagen, correspondiendo al \'area o volumen si se trata de una imagen binaria. 

Para el c\'alculo de los momentos centrales se agregan los componentes del centroide, $\overline{x}$ e $\overline{y}$, basados en los {\it raw moments}, como indican las ecuaciones \eqref{eq:avgFromRawMomts} y \eqref{eq:centralImageMoment}.

\begin{subequations}
\begin{equation}
    \label{eq:rawImageMoment}
    M_{pq} = \sum\limits_{x} \sum\limits_{y} x^p \cdot y^q \cdot f(x,y)
\end{equation}
\begin{equation}
    \label{eq:avgFromRawMomts}
    \overline{x} = \frac{M_{10}}{M_{00}}, \quad
    \overline{y} = \frac{M_{01}}{M_{00}}
\end{equation}
\begin{equation}
    \label{eq:centralImageMoment}
    \mu_{pq} = \sum\limits_{x} \sum\limits_{y} (x - \overline{x})^{p} \cdot (y - \overline{y})^{q} \cdot f(x,y)
\end{equation}
\end{subequations}

As\'i, es posible construir una matriz de covarianza, equivalente a la matriz hessiana en la ecuaci\'on \eqref{eq:HessianMat}, utilizando los momentos centrales de segundo orden, $\mu_{20}$, $\mu_{02}$ y $\mu_{11}$ divididos por el momento central de orden cero $\mu_{00}$ (ecuaciones \eqref{eq:mu20}, \eqref{eq:mu02} y \eqref{eq:mu11}), obteniendo los eigenvalores mediante la ecuaci\'on \eqref{eq:lambdaMoments}.

\begin{subequations}
\begin{align}
    \mu_{20}^{\prime} &= \frac{\mu_{20}}{\mu_{00}} = \frac{M_{20}}{M_{00}} - \overline{x}^{2} \label{eq:mu20} \\
    \mu_{02}^{\prime} &= \frac{\mu_{02}}{\mu_{00}} = \frac{M_{02}}{M_{00}} - \overline{y}^{2} \label{eq:mu02} \\
    \mu_{11}^{\prime} &= \frac{\mu_{11}}{\mu_{00}} = \frac{M_{11}}{M_{00}} - \overline{x}\cdot\overline{y} \label{eq:mu11}
\end{align}

\begin{equation}
    \label{eq:covMatLambda}
    cov[f(x,y)] = \begin{bmatrix}
        \mu_{20}^{\prime} & \mu_{11}^{\prime} \\
        \mu_{11}^{\prime} & \mu_{02}^{\prime} 
        \end{bmatrix}
\end{equation}

\begin{equation}
    \label{eq:lambdaMoments}
    \lambda_{1,2} = \dfrac{(\mu_{20}^{\prime} + \mu_{02}^{\prime}) \pm \sqrt{(\mu_{20}^{\prime} - \mu_{02}^{\prime})^{2} + 4\cdot \mu\prime_{11}^{2} }}{2}
\end{equation}
\end{subequations}

Con los valores de $\lambda$, es posible calcular caracter\'isticas de una estructura alargada como su excentricidad o su eje principal de inercia. Estas medidas pueden ayudar a mejorar la clasificaci\'on de segmentos del grafo durante la identificaci\'on de filamentos.


% informacion geometrica mediante calculo de angulos entre aristas
Un manera adicional de generar informaci\'on que facilite la discriminaci\'on de secciones del grafo es a trav\'es del c\'alculo de los \'angulos entre las aristas del grafo. Esto se relaciona a criterio de rectitud que tienen los filamentos, que var\'ia dependiendo de la c\'elula a la que pertenezca. Este comportamiento de los filamentos permite delimitar el \'angulo m\'aximo que 2 aristas contiguas pueden tener para ser considerados parte del mismo filamento, denomin\'andose este umbral como $Max\_Angle$. Cualquier valor por sobre $Max\_Angle$ permite descartar de forma absoluta esa combinanci\'on de aristas para un mismo filamento. 


A su vez, este criterio posee un segundo umbral, definido como $\theta$  que define el \'angulo m\'aximo bajo el que se considera que 2 aristas contiguas respetan con certeza la rectitud necesaria para formar parte del mismo filamento. Es decir, si 2 aristas contiguas forman un \'angulo en el rango $[0, \theta]$, deben ser parte del mismo filamento. El rango entre ambos umbrales, $]\theta, Max\_Angle]$ delimita los pares de aristas que a priori no representan combinaciones que respetan el criterio de rectitud, pero cuya explicaci\'on puede encontrarse en variaciones inducidas durante la extracci\'on del grafo desde la imagen, por lo que es necesario incorporar la exploraci\'on de estos pares de aristas.


Finalmente, para el caso de los nodos, el an\'alisis del grado de cada uno permite identificar la existencia de ciclos\cite{wilson1979introduction} en un filamento. La propiedad de un filamento de poder o no tener un ciclo es informaci\'on disponible priori que depende del tipo de c\'elula observada, permitiendo limitar posibles asociaciones entre nodos. En el caso particular de no permitir ciclos, un filamento no podr\'ia pasar m\'as de una vez por cada nodo que lo conforma. 

%se destaca dentro de los criterios: busca evitar perder información de la imagen/ampliar la informacion así como tener un costo computacional bajo, en conjunto con disminuir la interacción del usuario. 


% ambas opciones generan degeneraciones/deformaciones q afectan
\section{Exploraci\'on del Espacio de Soluciones}

La b\'usqueda de conjuntos de nodos o aristas adyacentes (denominados caminos) en un grafo, para individualizar uno o m\'as filamentos, constituye un espacio de soluciones que no es posible de recorrer en tiempo polinomial, dado que sin restricciones las combinaciones crecen exponencialmente\cite{buchin2007number}\cite{biswas2012hamiltonian}. Un planteamiento similar a lo anterior es el {\it Path Cover Problem} o PCP, en el que se descompone un grafo dirigido en caminos con el objetivo de obtener un conjunto de caminos. Cada nodo o arista debe pertenecer exactamente a un camino y los caminos pueden comenzar o terminar en cualquier parte del grafo.

En el caso de la individualizaci\'on de filamentos, es necesario que la definici\'on del problema considere como v\'alidos caminos que representen los casos de filamentos con o sin ciclos, as\'i como superposici\'on y/o cruce. Lo anterior impide forzar la pertenencia de un nodo o arista a un s\'olo camino. 

% explicar define: como lo hace define? y como se observa la potencial perdida de soluciones factibles al usar un solo criterio
La investigaci\'on de \cite{breuer2015define} (DeFiNe) descrita brevemente en el cap\'itulo \ref{chap:stateoftheart} presenta el {\it Filament Cover Problem} o FCP como una extensi\'on del PCP, flexibilizando la pertenencia de cada arista a al menos un camino. Adem\'as plantean como funci\'on objetivo la minimizaci\'on de la diferencia de la homogeneidad (aspereza) entre las aristas que componen un camino. Para acotar el espacio de soluciones, los autores de DeFiNe proponen 2 heur\'isticas fundamentando que el FCP en \'arboles en vez de grafos es soluble en tiempo polinomial. Aquello se basa en lo definido por \cite{lin2006vertex} que indica que para un problema de {\it covering}, existe un {\it Set System} $(S,C)$, donde S es el conjunto total y finito de sets, y C es un conjunto de subsets pertenecientes a S. En el caso espec\'ifico del {\it Minimum Set Cover}(SC), el objetivo es encontrar un subconjunto $C'$ de $C$ tal que cada elemento de $S$ pertenezca al menos 1 vez a uno de los miembros de $C'$.

%Un grafo completamente conectado puede tener n(n-1)/2 aristas, lo que para un n muy grande puede implicar un costo computacional alto
%A su vez, otro motivo para evitar que $G$ sea un grafo completo radica en que para los filamentos observados en la naturaleza no es una condici\'on comun... no encuentro la fuente de esto

%Denominamos todas estas opciones de caminos como el conjunto $P$, del cual debemos extraer un subconjunto $P'$ mediante una estrategia que permita realizar esto en tiempo polinomial independientemente de la cantidad de aristas del grafo.

Para un {\it set system} $(S,C)$ que pueda ser representando por un \'arbol $T$, es posible mutar la definici\'on de $S$ al conjunto de nodos que componen un grafo $G$, y a su vez, hacer que cada subset $c \in C$ represente un camino simple en $G$. Un camino simple es equivalente a un \'arbol simple (2 nodos son unidos por a lo m\'as 1 arista) y ac\'iclico.


Se destaca en \cite{lin2006vertex} que no todos los caminos simples de $G$ est\'an representados en $C$. Esta representaci\'on de {\it covering} de caminos, o {\it Path Cover}, es denominada {\it Vertex Covering by Paths on Graphs}(VcpG) y difiere de un {\it Path Cover} tradicional al permitir caminos que compartan nodos. Luego, para el caso de \'arboles, VcpG se renombra a VcpT ({\it Vertex Covering by Paths on Trees}), permitiendo el reemplazo de nodos por aristas sin generar cambios significativos en el planteamiento del problema. Esta modificaci\'on de nodos a aristas se denomina EcpT {\it Edge Covering by Paths on Trees}. %recordar mas adelante como posible perdida de soluciones factibles


Lo anterior establece el fundamento para que los autores de DeFiNe\cite{breuer2015define} utilicen EcpT como base, teniendo un m\'aximo de caminos $n(n)-1/2 = \mathcal{O}(n^{2})$, con una complejidad $\mathcal{O}(n^{4})$ para obtener esos caminos. La obtenci\'on de caminos a partir de un \'arbol $T$ que contiene los nodos del grafo $G$ se realiza mediante la divisi\'on continua del \'arbol en m\'ultiples bosques, hasta que los bosques resultantes sean s\'olo caminos simples. Sin embargo, los caminos resultantes no comparten nodos o aristas, es decir, no se superponen. Para solucionar aquello manteniendo la resoluci\'on del FCP en tiempo polinomial, se agrega en DeFiNe el par\'ametro $k$, que define el n\'umero m\'aximo de superposiciones de caminos en una arista. La complejidad con el nuevo par\'ametro queda como $\mathcal{O}(n^{2k+2})$.


%aca entran las formas de obtener esos arboles simples y bosques, con las heuristicas de define
Dado que el conjunto total de caminos en un grafo, definido como $P$, no es calculables en tiempo polinomial, DeFine propone 2 heur\'isticas para construir \'arboles de los cuales se puedan extraer un conjunto representativo de caminos simples, definidos como $P'$. Cada una de las heur\'isticas, {\it BFS} y {\it RMST}, explicadas en el cap\'itulo \ref{chap:stateoftheart}, proveen de un conjunto $P'$, al que se le aplica la funci\'on objetivo, para encontrar los miembros de $p \in P'$ que mejor minimicen la diferencia de homogeneidad en sus caminos. La restricci\'on de lo anterior es que cada arista del grafo pertenezca a lo menos a un camino. 

%\begin{equation}
%p = (e_1, e_2,..., e_n)\\
%p = ((v_1,v_2), (v_2,v_4),..., (v_n-1,v_n))
%\label{eq:path}
%\end{equation}

% Comparándose con define, Problema 1 caminos validos no llegan a P'

La generaci\'on del conjunto $P'$ planteado en DeFiNe puede llevar a excluir caminos v\'alidos al realizar la representaci\'on de un {\it set system} mediante un \'arbol, o al utilizar una sola propiedad asociada a un filamento en una de sus heur\'isticas. 


El presente trabajo plantea en la secci\'on \ref{sec:modeloOpti} un modelo de optimizaci\'on que permite generar caminos utilizando m\'as de una propiedad asociada a un filamento o al grafo que representa la red de filamentos. Esto permite individualizar filamentos en casos de superposici\'on, cruce o discriminando si el filamento puede tener o no ciclos. Adem\'as se presentan diversas heur\'isticas para reducir el tama\~no del espacio de soluciones.

%Ejemplo camino verde en Spinning Marchantia (imagenes)


\section{Modelo de Optimización}
\label{sec:modeloOpti}
%dado que se tienen muchos filamentos, se debe evaluar cual es mejor. Explicar como propiedades topológicas y geométricas tienen. Y como se ponderan para un peso que será minimizado o maximizado

En base a lo recopilado en las secciones previas de esta investigaci\'on, es posible destacar los siguientes aspectos al problema a resolver:

\begin{itemize}
    \item Se desconoce a priori el n\'umero de filamentos a buscar, dado que una imagen puede tener individualizaciones distintas para 2 expertos.
    \item Generalmente, se busca individualizar m\'as de un filamento por imagen, lo que conlleva a elegir los mejores filamentos entre las soluciones que se encuentren.
    \item El uso de un grafo para representar la red de filamentos puede implicar que las combinaciones de soluciones crezcan de manera exponencial.
\end{itemize}

Lo anterior implica que el problema de identificar filamentos a partir de un grafo puede ser clasificado como un problema de optimizaci\'on de restricciones\cite{blum2011hybrid}.

Un problema de optimización de restricciones, (COP por su sigla en ingl\'es) puede ser representado como $P = (S, \Omega, F)$, donde S es el espacio de soluciones, definido por un conjunto discreto de variables $X = 1 \dotsc n$, con valores $v_{i}^{j} \in D_{i} = \{v_{i}^{1} \dotsc  v_{i}^{|D_{i}|}\}$. Se define como una variable {\it instanciada} la asignaci\'on a $X_i$ de un valor $v_{i}^{j} \in D_i$. Una solución candidata $s \in S$ es una soluci\'on factible si satisface las restricciones del set $\Omega$. La funci\'on objetivo $F: S\rightarrow \mathbb R_{0}^{+}$, es la funci\'on de evaluaci\'on que asigna valores a las soluciones candidatas. Al mismo tiempo, se define $s^{*}$ como una soluci\'on \'optima y $S^{*}$ como un conjunto de soluciones \'optimas, relacionados mediante $s^{*} \in S^{*} \subseteq S $\cite{socha2008ant}.
Esta definici\'on permite aplicar la metaheur\'istica de optimizaci\'on basada en colonia de hormigas (ACO por su sigla en ingl\'es) a un modelo de un {\it COP}.
%COP es un CSP con función objetivo: https://en.wikipedia.org/wiki/Constrained_optimization#Constraint_optimization_problems

\subsection{Metaheur\'istica ACO}
El proposito de la metaheur\'istica ACO es encontrar una soluci\'on o un set de soluciones. Una soluci\'on $s$ consiste en un conjunto de componentes de soluci\'on $c_{ij} \in C, i = 1 \dotsc n, j = 1 \dotsc |D_i|$, por lo que una concatenaci\'on de componentes de soluci\'on forma el camino o {\it tour} que recorre una hormiga, desde un nodo o arista inicial hasta un nodo o arista final. La metaheur\'istica ACO se muestra en el algoritmo \ref{ACO-Algo}. En base a la definici\'on de variable {\it instanciada} del modelo COP, se tiene que la asignaci\'on $X_i = v_{i}^{j}$ es equivalente a seleccionar un componente $c_{ij}$ para una soluci\'on $s$ en ACO.

ACO consiste en un paso de inicializaci\'on y de tres componentes, las que no tienen un orden espec\'ifico: {\it Construccion\_de\_soluci\'on\_de\_cada\_hormiga(),  M\'etodo\_de\_b\'usqueda\_no\_local()} y {\it Actualizaci\'on\_de\_feromonas()}.


\begin{algorithm}[H]
\SetAlgoLined
\KwData{Variables $X_i \dotsc X_n$, dominios $D_1 \dotsc D_n$, Restricciones $\in \Omega$}
\KwResult{conjunto s\textquotesingle $ \subseteq S$ != $\emptyset$, si existen soluciones factibles}
 Ajuste de Par\'ametros \& inicializaci\'on de feromonas \;
 \While{Criterio de finalización no se cumple}{
   Planificaci\'on\_de\_Pasos\;{
   ~ Construccion\_de\_soluci\'on\_de\_cada\_hormiga()\;
   ~ M\'etodo\_de\_b\'usqueda\_no\_local() \% opcional (DaemonActions)\;
   ~ Actualizaci\'on\_de\_feromonas()\;
   }Fin\_Planificaci\'on\_de\_Pasos\;
 }
 \caption{Algoritmo Metaheur\'istica ACO}\label{ACO-Algo}
\end{algorithm}


\subsubsection{M\'etodo Construccion de soluci\'on de cada hormiga}
La elecci\'on de un componente $c_{ij}$ por una hormiga durante la construcci\'on de un camino,
se lleva a cabo mediante el c\'alculo de una probabilidad para cada componente $c_{ij}$ posible de elegir. Este conjunto de vecinos factibles se denomina $N(s^{P}) \subseteq C$. En la probabilidad de selecci\'on influye el camino ya escogido, denominado soluci\'on parcial $s^{P}$. Al comenzar un recorrido, cada hormiga es asignada una arista de acuerdo a la heur\'istica de asignaci\'on, la cual analiza 3 situaciones:
\begin{enumerate}
\item La arista a asignar debe tener al menos uno de sus nodos con grado 1, indicando que es el inicio o final de una parte del grafo.

\item De no haber aristas con esas caracter\'isticas disponibles, se realiza una asignaci\'on inicial de una arista con uno de sus nodos con grado 2 o superior, siempre que uno de los nodos sea la uni\'on de esta arista con otra con la que conformen un \'angulo en el rango $]\theta, Max\_Angle]$. 
%$\theta$ es un umbral que define el \'angulo m\'aximo, en grados, bajo el que se considera que 2 aristas contiguas respetan la rectitud necesaria para formar parte del mismo filamento. $Max\_Angle$ es un umbral que define el \'angulo m\'aximo, en grados, por sobre el cual se descarta de forma absoluta que 2 aristas contiguas forman parte del mismo filamentos. Este rango delimita los pares de aristas que a priori no representan combinaciones que respetan el criterio de rectitud, pero cuya explicaci\'on puede encontrarse en variaciones inducidas durante la extracci\'on del grafo desde la imagen, por lo que es necesario incorporar la exploraci\'on de estos pares de aristas.

\item De no existir aristas con alg\'un nodo que cumpla con los dos criterios previos, es posible asignar una arista aleatoria siempre que esta no forme parte de un camino recorrido por otra hormiga que haya sido evaluado como de buena calidad.
\end{enumerate}

% a diferencia de otros ACO, aca s^P != \emptyset al comienzo
Una vez asignada la primera arista seg\'un la heur\'istica previamente descrita, cada hormiga debe avanzar mediante la elecci\'on de nuevas aristas para a\~nadirlas a su recorrido. Este procedimiento se describe en la ecuaci\'on \eqref{eq:antProbabilities}, que corresponde a la probabilidad de elegir una arista a partir de un conjunto de aristas vecinas (conjunto $N(s^{P})$) dadas la aristas que ya pertenecen a la soluci\'on parcial de la hormiga ($s^P$), mediante la heur\'istica miope (ecuaci\'on \eqref{eq:heuristicaMiope}) que privilegia los candidatos que causen la menor desviaci\'on en la rectitud del camino. Se define que los componentes $c_{ij}$ que aportan con mayor probabilidad a la menor desviaci\'on son aquellos que en conjunto con el \'ultimo elemento elegido por la hormiga en ese punto ($c_{(i-1)j}$) forman un \'angulo en el rango $[0, \theta]$. El criterio de finalizaci\'on para la hormiga corresponde a que el conjunto $N(s^{P}) = \emptyset$.

%P(C_{ij} | s^{P}) = P_{n_{i},n_{j}} = P_{e_{x}}
\begin{equation}
P(c_{ij} | s^{P}) = \frac
        {\tau_{ij}^{\alpha} \cdot \eta_{ij}^{\beta}}
        {\sum\limits_{c_{ij}\in N(s^p)}{\tau_{ij}^{\alpha} \cdot \eta_{ij}^{\beta} } }, \forall c_{ij} \in N(s^{P})
\label{eq:antProbabilities}
\end{equation}

Otro aspecto de la ecuaci\'on \eqref{eq:heuristicaMiope} radica en la posibilidad de elecci\'on de elementos $c_{ij}$ que tienen un \'angulo en el rango $]\theta, \text{Max\_Angle}]$ con el componente de soluci\'on $c_{(i-1)j}$. Esto facilita la exploraci\'on de soluciones/caminos que de forma miope aparecen como de calidad no \'optima, y que para los efectos de este trabajo se denominan como de {\it calidad intermedia}. Esta evaluaci\'on consistente en disminuir la probabilidad de elecci\'on a medida que la diferencia entre el \'angulo que forman $c_{ij}$ y $c_{(i-1)j}$  , y la mitad de $\theta$ se incrementa.


%donde cada uno representa a una arista en esta investigaci\'on, y . Al momento de que la diferencia sea 90\textdegree, la probabilidad de asignaci\'on se reduce al 50\% de la probabilidad de un componente $c_{ij} \in [0, \theta]$.

\begin{equation}
    \eta_{ij} = 
        \begin{cases} 
        \text{Max\_Score si } \measuredangle(c_{ij}, c_{(i-1)j}) \in [0, \theta]\\[3ex]
        
        \text{Max\_Score} \cdot \left(1 - \dfrac{ \left| \measuredangle(c_{ij}, c_{(i-1)j}) - \frac{\theta}{2} \right|} {180} \right)  \text{ si } \measuredangle(c_{ij}, c_{(i-1)j}) \in \quad ]\theta, \text{Max\_Angle}].\\[3ex]
        
        \text{0 en otro caso;}
        \end{cases}
    \label{eq:heuristicaMiope}
\end{equation}

Con la finalidad de cuantificar la calidad de la soluci\'on construida por una hormiga, en cada selecci\'on de arista realizada, se suma el resultado de la heur\'istica miope al valor que la calidad de la hormiga lleva hasta ese punto. Cada hormiga comienza con una calidad 0, que al finalizar el {\it tour} es dividida por el n\'umero de aristas menos 1, para normalizar. Se establece que para ser considerada una soluci\'on de buena calidad, la hormiga debe tener una calidad mayor o igual a $\frac{Max\_Score}{2}$.
    
\subsubsection{M\'etodo de b\'usqueda no local}
Una vez que la hormiga termina un {\it tour}, es posible agregar un m\'etodo de {\it feedback} sobre la calidad del recorrido realizado, basado en l\'ogicas globales/centralizadas que escapan de la b\'usqueda local que realiza cada hormiga. Estos m\'etodos, denominados {\it Daemon Actions} en ingl\'es, permiten en el caso de una metaheur\'istica ACO gen\'erica seleccionar las hormigas de mejor calidad para incrementar las feromonas m\'as alla de lo que la {\it Actualizaci\'on\_de\_feromonas()} lo hace. 

Para la individualizaci\'on de filamentos, la evaluaci\'on global corresponde a eliminar soluciones candidatas que no aporten informaci\'on nueva. A modo de ejemplo, si $s_a$ y $s_b$ son las soluciones de las hormigas $a$ y $b$ respectivamente y cumplen con las siguientes condiciones:

\begin{itemize}
    \item $\forall c_{ij} \in s_a \in [0, \theta]$ y $\forall c_{ij} \in s_b \in [0, \theta]$
    \item $\forall c_{ij} \in s_a$ fueron electos por la hormiga $a$ con $P(c_{ij} | s_{a}^{P}) = 1$ y $\forall c_{ij} \in s_b$ fueron electos por la hormiga $b$ con $P(c_{ij} | s_{b}^{P}) = 1$
    \item $s_a \subseteq s_b$
\end{itemize}

Se tiene que $s_a$ no aporta m\'as informaci\'on que $s_b$, por lo que $s_a$ puede descartarse. Se denomina a $s_b$ como un segmento, el cual se comporta como una secci\'on indivisible de filamento. Este m\'etodo se ejecuta al comparar dos soluciones 
%RIESGO ASOCIADO a overmatch!!

%Si dos soluciones, $s_i$ y $s_j$ de las hormigas $i$,$j$, conformadas solamente por componentes $c_{ij} \in [0, \theta]$  (todos los componentes son de {\it buena calidad}), y que adem\'as eran la \'unica opci\'on posible en cada avance de la hormiga (probabilidad 1 de ser elegidas)
%tal que $s_i \subseteq s_j$ o viceversa, se tiene que una soluci\'on candidata no aporta nueva informaci\'on.
    
\subsubsection{M\'etodo Actualizaci\'on de feromonas}
\label{subsec:pheroUpdate}
Una vez que la hormiga termina un {\it tour}, esta debe actualizar las feromonas ($\tau_{ij}$) asociadas a los componentes de soluci\'on que la conforman. En una metaheur\'istica ACO tradicional, se aumenta el valor en los $c_{ij}$ que construyen un camino de buena calidad, mientras que debe realizar lo contrario para los componentes de soluci\'on que son parte de un recorrido de mala calidad. Adem\'as, los valores de las feromonas sufren decaimiento en el tiempo, dado por el par\'ametro $\rho$, que busca evitar la convergencia que las feromonas pueden causar en caminos de buena soluci\'on obtenidos al inicio de las iteraciones.


En la individualizaci\'on de filamentos, se utilizan {\it anti-feromonas}, con el proposito de indicar a las hormigas de futuras iteraciones cuales combinaciones de aristas/componentes de soluci\'n no llevan a resultados de buena calidad. Esto se fundamenta en la utilidad de las {\it anti-feromonas} para acotar el espacio de soluciones $S$. La forma de usar las {\it anti-feromonas} corresponde a {\it Substractive Anti-Pheromone} (SAP por su sigla en ingl\'es), la que introduce el par\'ametro $\gamma$ como factor de reducci\'on/penalizaci\'on. \cite{montgomery2002anti} indica que con $\gamma = 0.5$ SAP obtiene los mejores resultados. Por otra parte, el par\'ametro $\rho$ utilizado en las feromonas tradicionales no se utiliza en SAP.


Una modificaci\'on que se introduce en este trabajo con respecto a la aplicaci\'on de  feromonas y anti-feromonas es que estas com\'unmnente s\'olo se encuentran asociadas al componente $c_{ij}$ respectivo. Como variaci\'on para la individualizaci\'on/reconocimiento de filamentos, se propone asociar el valor de la anti-feromona no s\'olo con el componente $c_{ij}$, sino que adem\'as con uno o m\'as de los elementos que fueron elegidos en pasos anteriores por la hormiga que los selecciona. La explicaci\'on de lo anterior se basa en la posbilidad de desglosar una soluci\'on $s$ en segmentos $seg_{n}$ donde $n$ se\~nala el n\'umero de segmento al que corresponde dentro de la soluci\'on $s$. El segmento $seg_1$ comienza con la primera arista/componente $c_{ij}$ asignada a la hormiga de acuerdo a la heur\'istica de asignaci\'on, y termina en el primer elemento $c_{ij}$ con el que forma un \'angulo de calidad intermedia (sin incluirlo), siendo este mismo elemento el que inicia el segmento siguiente.
%Existen $N + 1$ segmentos en $s$ si la soluci\'on contiene $N$ elementos $c_{ij} \in ]\theta, 90]$ (de calidad intermedia). 


Luego, mediante la anti-feromona se relaciona un segmento $seg_n \subset s$ y la componente de soluci\'on $c_{ij} \in seg_{n+1} \in s$, cuya combinaci\'on es parte de una soluci\'on de mala calidad, con el fin de evitar que otras hormigas que pasen por $seg_n$ elijan $c_{ij}$. El rol de la anti-feromona es penalizar la probabilidad de elecci\'on de $c_{ij}$ para las hormigas que tengan al segmento $seg_n$ en su soluci\'on parcial $s^{P}$. Se debe destacar que esto es necesario ya que si s\'olo se utiliza la anti-feromona para penalizar $c_{ij}$, se puede ocasionar la perdida de capacidad de exploraci\'on de hormigas que provengan de otros recorridos parciales distintos a $seg_n$. La perdida de exploraci\'on tambi\'en sucede en el caso que el segmento $seg_{n+1}$ contenga solo 1 arista y no sea el segmento en el que finaliza el recorrido de la hormiga. Para subsanar aquel caso, a este tipo de segmentos se le a\~naden los nodos del segmento que lo precede, $seg_{n}$, con el objetivo de evitar que el an\'alisis del par $seg_{n+1}$ con la componente $c_{ij}$ que inicia el segmento $n+2$ sea s\'olo de 2 aristas.


Adicionalmente a lo anterior, se ha agregado un l\'imite de 2 penalizaciones como m\'aximo para cada anti-feromona. Al alcanzar este l\'imite, se reduce el valor de la anti-feromona a 0, haciendo imposible la elecci\'on de la componente de soluci\'on $c_{ij}$ para las hormigas cuya soluci\'on parcial $s^{P}$ contenga el segmento en el par componente,segmento penalizado.


La anti-feromona se aplica sobre hormigas que han finalizado su recorrido y cuya calidad normalizada sea menor a {\it Max\_Score}, ya que esto implica que al menos 1 de las aristas del recorrido es de calidad intermedia, necesitando un an\'alisis adicional para determinar si corresponde a una soluci\'on de buena calidad. El an\'alisis adicional consiste en evaluar la curvatura del recorrido, as\'i como la magnitud del desplazamiento entre la proyecci\'on de un segmento en relaci\'on a otro segmento contiguo. 

La curvatura de recorrido de una hormiga $a$ es el \'angulo formado por el nodo inicial ($n_{a1}$), el centro de masa de la misma hormiga ($mc_{a}$) y el nodo final ($n_{af}$). Este \'angulo no debe superar el umbral definido al multiplicar el \'angulo $\theta$ por un factor denominado {\it Max\_Axial\_Displacement}. Este factor permite flexibilizar la tolerancia de la curvatura en base a $\theta$. Si el recorrido de la hormiga tiene un \'angulo igual o mayor al umbral, implica que la soluci\'on encontrada es demasiado curva para representar un filamento, por lo que se penaliza el par $\langle c_{ij}$,$ seg_{n}\rangle$ donde $c_{ij}$ es la componente de soluci\'on/arista que inicia el \'ultimo segmento de la hormiga, mientras que $seg_{n}$ es el segmento que lo precede. Posterior a la penalizaci\'on, se desecha la soluci\'on.

El criterio de curvatura se refleja en la ecuaci\'on \eqref{eq:antiPheroSAP_Angle}.
%\begin{equation}
%    \label{eq:antiPheroSAP_Angle}
%    \tau_{ij} \leftarrow \tau_{ij} \cdot \gamma \quad \forall \langle c_{ij},seg_{n}\rangle > \textrm{Max\_Axial\_Displacement}
%\end{equation}

\begin{equation}
    \tau_{ij} \leftarrow
        \begin{cases}
        \tau_{ij} \cdot \gamma \text{ si } \measuredangle((n_{a1}, mc_{a}), (mc_{a}, n_{af})) < \theta \cdot \text{Max\_Axial\_Displacement}\\[3ex]
        
        \text{0 si } \tau_{ij} \leq 0.25 \\[3ex]
        \tau_{ij} \quad \text{en otro caso}
        \end{cases}
    \label{eq:antiPheroSAP_Angle}
\end{equation}

%agregar referencia? tindemans rod straightness \cite{hawkins2010model} of MTs o 
El an\'alisis respecto a la magnitud del desplazamiento entre la proyecci\'on de un segmento en relaci\'on a los segmentos que lo preceden se fundamenta en la rigidez que algunos tipos de filamentos, como los microt\'ubulos y los filamentos de actina poseen\cite{stam2017filament}. El criterio de rigidez consiste en analizar los segmentos con respecto a la totalidad de sus predecesores, comenzando por el \'ultimo segmento recorrido por la hormiga, es decir, desde el extremo final de la soluci\'on. A partir de cada par de segmento-predecesores, $\langle seg_{n}$,$seg_{1,n-1}\rangle$, se debe seleccionar el miembro del par de mayor longitud, definido como $s_{max} = \max(\norm{seg_{n}}, \norm{seg_{1,n-1}})$, para calcular el \'angulo suplementario que forma con respecto al otro miembro del par, definido como $s_{min}$. El \'angulo suplementario, $\measuredangle supl(s_{max},s_{min})$ es el equivalente a calcular el \'angulo entre la proyecci\'on de $s_{max}$ y $s_{min}$.

Luego $s_{min}$ es multiplicado por el seno del \'angulo suplementario, estableciendo el desplazamiento con respecto al eje que forma $s_{max}$ y su proyecci\'on. El umbral que delimita al desplazamiento calculado previamente se define como el m\'ultiplo de $s_{max}$ por 10\% de {\it Max\_Axial\_Displacement}. 

Si el \'angulo suplementario es menor a $\theta$, se puede declarar que se cumple el criterio de rigidez. Lo anterior se refleja en la ecuaci\'on \eqref{eq:antiPheroSAP_Axial}. 


%Esta propiedad puede ser utilizada para delimitar como un segmentos de una hormiga se relacionan con el resto de la soluci\'on, ya que estos ser\'ia un reflejo indirecto de los movimientos din\'amicos de un filamento en el tiempo, capturados en un punto a trav\'es de una imagen. 
%La rigidez de un filamento puede describirse mediante la relaci\'on que existe entre un segmento de una hormiga, con respecto a todos los segmentos que lo preceden. 
\begin{equation}
    \tau_{ij} \leftarrow
        \begin{cases}
        \begin{split}
         \tau_{ij} \cdot \gamma \text{ si } & \sin(\measuredangle supl(s_{max},s_{min}) > s_{min} \cdot 0.1 \cdot \text{Max\_Axial\_Displacement} \\ & \land \measuredangle supl(s_{max},s_{min}) > \theta    
        \end{split}
        \\[3ex]
        
        \text{0 si } \tau_{ij} \leq 0.25 \\[3ex]
        \tau_{ij} \quad \text{en otro caso}
        \end{cases}
    \label{eq:antiPheroSAP_Axial}
\end{equation}

Si ambas evaluaciones son completadas correctamente, se declara a la soluci\'on como de buena calidad.

%Finalmente, la ecuaci\'on \eqref{eq:antiPheroSAP} refleja la aplicaci\'on de las {\it anti-feromonas} sobre el par $\langle c_{ij}$,$ seg_{n}\rangle \forall c_{ij} \in ]\theta, Max\_Angle]$  donde $c_{ij} \in seg_{n+1}$, y cuya elecci\'on dio lugar a $seg_{n+1}$ que se aleja del desplazamiento axial m\'aximo que un filamento puede soportar.

En relaci\'on a los dominios $D_i$ declarados en la definici\'on del modelo de un COP, se destaca que en el caso de la individualizaci\'on de filamentos existe solo $D_1 \in D$, ya que la instanciaci\'on de variables ($X_i = v_{i}^{j}$ o $c_{ij}$) tiene una sola asignaci\'on posible, lo que lleva a una simplificaci\'on del componente $j$ en las ecuaciones presentadas.

\subsubsection{Inicializaci\'on de la metaheur\'istica ACO}

En el paso de inicializaci\'on de ACO se deben definir los valores de los par\'ametros relacionados a las feromonas y las heur\'isticas utilizadas. Para las feromonas se configura el valor inicial de $\tau_{ij}$ en 1 para los pares $\langle c_{ij}$,$ seg_{n}\rangle \> \forall c_{ij} \in \> ]\theta, Max\_Angle]$, dado que al usar SAP esta probabilidad se ir\'a reduciendo de acuerdo a un factor $\gamma$ seg\'un lo explicado en la secci\'on \ref{subsec:pheroUpdate}. El valor de $\gamma$ se define en 0.5, en base a lo encontrado en la literatura.

En el caso de los par\'ametros utilizados en diversas partes del modelo de optimizaci\'on, $\theta$ y {\it Max\_Axial\_Displacement} se encuentran asociados al tipo de c\'elula que se observa, lo que debe ser indicado como informaci\'on a priori. Los valores de $\theta$ son de 30\textdegree ~para microt\'ubulos de planta y neuronas. Por su parte, {\it Max\_Axial\_Displacement} recibe un valor de 1.5 para el caso de los microt\'ublos de planta o de 2.5 para las neuronas. Existen otras opciones de c\'elulas disponibles en la implementaci\'on del modelo de optimizaci\'on. El par\'ametro {\it Max\_Angle} se define como el m\'aximo entre 2.5 veces $\theta$ y 90. El par\'ametro {\it Max\_Score} utilizado por la heur\'istica miope se fija en 2.


Por su parte, el criterio de finalizaci\'on consiste en generar nuevas hormigas hasta que todas las aristas sean parte de al menos una soluci\'on de buena calidad, o que el n\'umero de hormigas generadas sea superior a 4 veces la cantidad de aristas.


En resumen, las condiciones del problema de identificaci\'on de filamentos dan pie a establecer su representaci\'on mediante un problema de optimizaci\'on de restricciones (COP), estando el modelo para la resoluci\'on del COP basado en la metaheur\'istica ACO para su resoluci\'on. La implementaci\'on del modelo de optimizaci\'on presentado es un algoritmo denominado {\it Phil} el que se encuentra escrito en C++.